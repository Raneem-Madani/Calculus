\setcounter{chapter}{9}
\chapter{Parametric Equation and Polar Coordinate}
\section{Parametric Equation}
\noindent
\hfill
\begin{minipage}{0.3\textwidth}
$$x=1\Rightarrow y=1$$
$$x=?\Rightarrow y=?$$
$$y=f(x)$$
\end{minipage}
\begin{minipage}{0.3\textwidth}
\includegraphics[width=\linewidth]{1.jpg}
\end{minipage}
\begin{minipage}{0.3\textwidth}
$$y=t+1$$
$$x=1-t^2$$
$$t=1\Rightarrow $$
$$x=0~,y=2,(0,2)$$
\end{minipage}
\noindent{\color{smalt(darkpowderblue)}\rule{\linewidth}{.2mm}}
%-----------------------------------------------------------------------------------
\begin{example}
Sketch the curve define by the \textbf{\color{smalt(darkpowderblue)} Parametric equation}\\
$x=t^2-2t~,~y=t+1.$\\
\underline{\textbf{\large}\color{smalt(darkpowderblue)}Solution} \\
\begin{minipage}{0.25\textwidth}
	\begin{tabular}{ |c | c | c | }
	\hline
	t & x & y \\
{\color{red}2} & {\color{red}0} & {\color{red}3}\\
{\color{red}1} & {\color{red}-1} & {\color{red}2}\\
		0 & 0 & 1\\
{\color{red}-1} & {\color{red}3}& {\color{red}0}\\
{\color{red}-2} & {\color{red}8}& {\color{red}-1}\\
        \vdots &\vdots &\vdots
	\end{tabular}
\end{minipage}
\begin{minipage}{0.35\textwidth}
{\color{smalt(darkpowderblue)}$$t=y-1$$
$$\Downarrow$$
$$x=(y-1)^2-2(y-1)$$
$$=y^2-2y+1-2y+2$$
$$x=y^2-4y+2$$}
\end{minipage}
\begin{minipage}{0.6\textwidth}
\includegraphics[width=5cm]{2.jpg}
\end{minipage}
\end{example}
%----------------------------------------------------------------------------------
\begin{example}
Sketch $x=t^2-2t~,~y=t+1~,~-1\leq x\leq 2.$\\
\underline{\textbf{\large}\color{smalt(darkpowderblue)}Solution} \\
\begin{minipage}{0.6\textwidth}
$${\color{smalt(darkpowderblue)}Parametric~equation:}$$
$$x=f(t)~,~y=g(t)$$
$$a\leq t\leq b$$
$$Initial~point~(f(a),g(a))$$
$$Terminal~point~(f(b),g(b))$$
\end{minipage}
\noindent\begin{minipage}{0.5\textwidth}
\includegraphics[width=5cm]{3.jpg}
\end{minipage}
\hfill
\end{example}
\noindent{\color{smalt(darkpowderblue)}\rule{\linewidth}{.2mm}}
%%%%%%%%%%%%%%%%%%%%%%%%%%%%%%%%%%%%%%%%%%%%%%%%%%%%%%%%%%%%%%%%%%%%%%5
\begin{example}
Sketch $x=\cos t~,~y=\sin t$\\
\underline{\textbf{\large}\color{smalt(darkpowderblue)}Solution}\\
\begin{minipage}{0.5\textwidth}
$$\cos^2 t+\sin^2 t=1$$
$$x^2+y^2=1$$
$$t=0\Rightarrow (1,0)$$
$$t=\cfrac{\pi}{2}\Rightarrow (0,1)$$
\end{minipage}
\begin{minipage}{0.5\textwidth}
\includegraphics[width=5cm]{4.jpg}
\end{minipage}
\end{example}
%------------------------------------------------------------------------
\noindent{\color{smalt(darkpowderblue)}\rule{\linewidth}{.2mm}}
%-------------------------------------------------------------
\begin{example}
Sketch $x=\cos 2t~,~y=\sin 2t$\\
\underline{\textbf{\large}\color{smalt(darkpowderblue)}Solution} \\
\begin{minipage}{0.5\textwidth}
$$\cos^2 2t+\sin^2 2t=1$$
$$x^2+y^2=1$$
$$t=0\Rightarrow (0,0)$$
$$t=\cfrac{\pi}{2}\Rightarrow (-1,0)$$
\end{minipage}
\begin{minipage}{0.5\textwidth}
\includegraphics[width=5cm]{4.jpg}
\end{minipage}
\end{example}
%----------------------------------------------------------------------------------
\begin{exercise}
Sketch $x=\cos\cfrac{1}{2}t~,~y=\sin\cfrac{1}{2}t$ 
\end{exercise}
%----------------------------------------------------------------------------------

\begin{example}
Sketch $x=2\cos t~,~y=2\sin t$\\
\underline{\textbf{\large}\color{smalt(darkpowderblue)}Solution} \\
\begin{minipage}{0.5\textwidth}
$$x^2+y^2=4\cos^2 t+4\sin^2 t$$
$$x^2+y^2=4$$
\end{minipage}
\begin{minipage}{0.5\textwidth}
\includegraphics[width=5cm]{5.jpg}
\end{minipage}
\end{example}
\noindent{\color{smalt(darkpowderblue)}\rule{\linewidth}{.2mm}}

%---------------------------------------------------------------------------------
\begin{example}
Sketch $x=2\cos t+1~,~y=2\sin t+2$\\
\underline{\textbf{\large}\color{smalt(darkpowderblue)}Solution} \\
\begin{minipage}{0.48\textwidth}
$$x=2\cos t+1\Rightarrow x-1=2\cos t$$
$$y=2\sin t+2\Rightarrow y-2=s\sin t$$
$$(x-1)^2+(y-2)^2=4(\cos^2 t+\sin^2 t)$$
$$(x-1)^2+(y-2)^2=4$$
\end{minipage}
\begin{minipage}{0.18\textwidth}
\includegraphics[width=5cm]{6.jpg}
\end{minipage}
\begin{minipage}{0.5\textwidth}
$$x=r\cos t+h$$
$$y=r\sin t+k$$
$$circle~center~(h,k)$$
$$radius~r$$
\end{minipage}
\end{example}

\noindent{\color{smalt(darkpowderblue)}\rule{\linewidth}{.2mm}}
%----------------------------------------------------------------------------------
\begin{example}
Sketch $x=2\cos t~,~y=3\sin t$\\
\underline{\textbf{\large}\color{smalt(darkpowderblue)}Solution}\\
\begin{minipage}{0.5\textwidth}
$$x=2\cos t\Rightarrow \cos t=\cfrac{x}{2}$$
$$y=3\sin t\Rightarrow\sin t=\cfrac{y}{3}$$
$$\Rightarrow (\cfrac{x}{2})^2+(\cfrac{y}{3})^2=1$$
\end{minipage}
\begin{minipage}{0.5\textwidth}
\includegraphics[width=5cm]{5.jpg}
\end{minipage}
\end{example}
%----------------------------------------------------------------------------------
\begin{example}
Sketch $x=\sin t~,~y=\sin^2 t$\\
\underline{\textbf{\large}\color{smalt(darkpowderblue)}Solution} \\
\begin{minipage}{0.5\textwidth}
$$y=x^2$$
$$t=0\Rightarrow(0,0)$$
$$t=\cfrac{\pi}{2}\Rightarrow(1,1)$$
$$t=\pi\Rightarrow(0,0)$$
$$t=\cfrac{3}{2}\pi\Rightarrow(-1,1)$$
$$t=2\pi\Rightarrow(0,0)$$
\end{minipage}
\begin{minipage}{0.5\textwidth}
\includegraphics[width=5cm]{7.jpg}
\end{minipage}
\end{example}
\noindent{\color{smalt(darkpowderblue)}\rule{\linewidth}{.2mm}}
%----------------------------------------------------------------------------------
\begin{example}
$\boxed{\cfrac{37}{628}}$\begin{enumerate}
    \item[(a)] $x=t^3~,~y=t^2$, {\color{red}$\Rightarrow t=x^\frac{1}{3}\Rightarrow y=x^\frac{2}{3}$}
    \item[(b)] $x=t^{6}~,~y=t^{4}$, {\color{red}$t=x^{\frac{1}{6}}\Rightarrow y=x^\frac{4}{6}\Rightarrow y=x^\frac{2}{3}$}
    \item[(c)] $x=e^{-3t}~,~y=e^{-2t}$, {\color{red}$\Rightarrow x=(e^{-t})^3\Rightarrow e^{-t}=x^\frac{1}{3}\Rightarrow y=(e^{-t})^2=(x^\frac{1}{3})^2\Rightarrow 
    y=x^\frac{2}{3}$}
    \end{enumerate}
    \underline{\textbf{\large}\color{smalt(darkpowderblue)}Solution} \\
    \begin{minipage}{0.1\textwidth}
$$(a) ~ x=t^6$$
$$y=t^4$$
$$y=x^\frac{2}{3}$$\\
\end{minipage} 
\begin{minipage}{0.62\textwidth}
$$(b)  x=t^6$$
$$y=t^4$$
$$y=x^\frac{2}{3}$$\\
\end{minipage} 
\begin{minipage}{0.18\textwidth}
$$(c)~x=e^{-3t}=\cfrac{1}{e^{3t}}$$
$$y=e^{-2t}=\cfrac{1}{e^{2t}}$$
$$y=x^\frac{2}{3}$$\\
\end{minipage}\\ 
\noindent\begin{minipage}{0.3\textwidth}
\includegraphics[width=4.2cm]{h1.jpg}
$$t=-1 \Rightarrow (-1,1) $$
$$t=0 \Rightarrow (0,0)$$
$$t=1 \Rightarrow (1,1)$$
\end{minipage}
\noindent\begin{minipage}{0.4\textwidth}
\includegraphics[width=4.2cm]{h2.jpg}
$$t=-1 \Rightarrow (1,1) $$
$$t=0 \Rightarrow (0,0)$$
$$t=1 \Rightarrow (1,1)$$
\end{minipage}
\noindent\begin{minipage}{0.3\textwidth}
\includegraphics[width=4.2cm]{h3.jpg}
$$t=-1 \Rightarrow (e^3,,e^2) $$
$$t=0 \Rightarrow (1,1)$$
$$t=1 \Rightarrow (\cfrac{1}{e^3},\cfrac{1}{e^2})$$
\end{minipage}
\end{example}

%%%%%%%%%%%%%%%%%%%%%%%%%%%%%%%%%%%%%%%%%%%%%%%%%%%%%%%%%%%%%%%%%%%%-----------------------
\underline{\large\textbf{\color{smalt(darkpowderblue)}{Cycloid}}}
\noindent\begin{minipage}{0.3\textwidth}
\includegraphics[width=4cm]{h5.jpg}
\end{minipage}
\noindent\begin{minipage}{0.3\textwidth}
\includegraphics[width=4cm]{h4.jpg}
\end{minipage}
\noindent\begin{minipage}{0.3\textwidth}
\includegraphics[width=4cm]{h6.jpg}
\end{minipage}\\
\hspace*{1.8cm}$$\underbrace{x=r(\theta-sin\theta)~~y=r(1-cos\theta)}~ ~0 \leq \theta \leq 2*\pi$$
$$Cycloid$$\\
\noindent{\color{smalt(darkpowderblue)}\rule{\linewidth}{.2mm}}
%%%%%%%%%%%%%%%%%%%%%%%%%%%%%%%%%%%%%%%%%%%%%%%%%%%%%%%%%%%%%%%%%%%
\begin{problem}
1, 3, 4, 7, 8, 10, 12, 15, 16, 25, 26, 37.
\end{problem}
%%%%%%%%%%%%%%%%%%%%%%%%%%%%%%%%%%%%%%%%%%%%%%%%%%%%%%%%%%%%%%%%%%%%-
\section{Calculus with Parametric Equations.}
let $x=f(t) , y=g(t)$   Then \\ $$\cfrac{dy}{dx}=\cfrac{\cfrac{d(y)}{d(t)}}{\cfrac{d(x)}{d(t)}}=\cfrac{g'(t)}{f'(t)}\hspace{.5cm} ,f'(t)\neq 0$$
%%%%%%%%%%%%%%%%%%%%%%%%%%%%%%%%%%%%%%%%%%%%%%%%%%%%%%%%%%%%%%%%%%%%
\begin{example}
if $x=cost , y=sint$\hspace{1cm}\\
\underline{\textbf{\large}\color{smalt(darkpowderblue)}Solution} \\
$$\cfrac{dy}{dx}=-\cfrac{cost}{sint}=-cott$$
  \begin{equation}
\begin{split} \cfrac{d^2y}{dx^2} &=\cfrac{d}{dx} \cfrac{dy}{dx} = \cfrac{\cfrac{d}{dt}\cfrac{dy}{dx}}{\cfrac{dx}{dt}} \\
&= \cfrac{csc^2t}{-sint} \\
&= -csc^3t
\end{split}
\end{equation}
\end{example}
%%%%%%%%%%%%%%%%%%%%%%%%%%%%%%%%%%%%%%%%%%%%%%%%%%%%%%%%%%%%%%%%%%%%-
\noindent{\color{smalt(darkpowderblue)}\rule{\linewidth}{.2mm}}
\begin{example}
if $x=t^3 , y=3t$, Find $\cfrac{d^2y}{dx^2}$\\
\underline{\textbf{\large}\color{smalt(darkpowderblue)}Solution} \\
\noindent\begin{minipage}{0.3\textwidth}
$$\cfrac{dy}{dx}=\cfrac{3}{3t^2}=t^{-2}$$
$$\cfrac{d^2y}{dx^2}=\cfrac{\cfrac{d}{dt}\cfrac{dy}{dx}}{\cfrac{dx}{dt}}$$ 
$$\cfrac{-2t^{-3}}{3t^2} = \cfrac{-2}{3}t^{-5}$$
$$\cfrac{d^2y}{dx^2}=\cfrac{-2}{3}x^\frac{-5}{3}$$
\end{minipage}
\noindent\begin{minipage}{0.6\textwidth}
$$y=3x^\frac{1}{3}$$
$$\cfrac{dy}{dx}=x^\frac{-2}{3}$$
\color{smalt(darkpowderblue)}{$$\cfrac{d^2y}{dx^2}=\cfrac{-2}{3}x^\frac{-5}{3}$$}
$\swarrow$
\end{minipage}
\end{example}
%%%%%%%%%%%%%%%%%%%%%%%%%%%%%%%%%%%%%%%%%%%%%%%%%%%%%%%%%%%%%%%%%%%%-
\begin{exercise}
if $y=cost+t ,\\ x=1-t+t^2$ \hspace{5mm}~Find $\cfrac{d^2y}{dx^2}$ 
\end{exercise}
%%%%%%%%%%%%%%%%%%%%%%%%%%%%%%%%%%%%%%%%%%%%%%%%%%%%%%%%%%%%%%%%%%%%
\begin{example}
Consider the parametric equations : $x=t^2 , y=t^3-3t$
\begin{enumerate}
    \item[(a)] Show that hte curve has two tangents at (3,0) ,Find the tangent line
    \item[(b)]Find where the tangent is horizontal ? Vertical ?
    \item[(c)]Find where the curve is concave up? down ?
    \item[(d)]Sketch the curve .
\end{enumerate}
\underline{\textbf{\large}\color{smalt(darkpowderblue)}Solution} \\
(a)~$(3,0) \Rightarrow ~x=3 ~, y=0$\\$t^2=3 \hspace{3cm} t^3-3t=0$ $\\ \downarrow \hspace{4cm} \downarrow \\ t=\sqrt{3} , -\sqrt{3} \hspace{2cm} t=0 ,\sqrt{3} , -\sqrt{3} $ 
\hspace{1cm} {\color{red}Refuse t=0} \\
The curve passes the point (3,0) two times at $t=\sqrt{3} , -\sqrt{3} $ \\ 
$\cfrac{dy}{dx}=\cfrac{\cfrac{dy}{dt}}{\cfrac{dx}{dt}} = \cfrac{3t^-3}{2t}$ \\ 
at $t=\sqrt{3}\Rightarrow \cfrac{dy}{dx}=\cfrac{3\sqrt{3}^2-3}{2\sqrt{3}}=\cfrac{3}{\sqrt{3}}=\sqrt{3}$\\
at $t=-\sqrt{3}\Rightarrow \cfrac{dy}{dx}=\cfrac{3*3-3}{-2\sqrt{3}}=-\sqrt{3}$ \\
So equation (1) \\
$y-0=\sqrt{3}(x-3)$\\
And equation (2) \\
$y-0=-\sqrt{3}(x-3)$\\
\end{example}
\noindent{\color{smalt(darkpowderblue)}\rule{\linewidth}{.2mm}}
\textbf{\color{smalt(darkpowderblue)}\large \underline{Area:}}
If $y=g(t)~,~x=f(t)$\\\\
\begin{minipage}{0.5\textwidth}
\begin{itemize}
    \item Then the area between the curve $C$ and $x-axis$ is 
    $${\color{smalt(darkpowderblue)}A=\int_a^b g(t)f'(t)dt.}$$
    \item The area with the $y-axis$ is 
    $${\color{smalt(darkpowderblue)}A=\int_a^b f(t)g'(t)dt.}$$
\end{itemize}
\end{minipage}
\begin{minipage}{0.5\textwidth}
\includegraphics[width=6cm]{1a.jpg}
\end{minipage}
\noindent{\color{smalt(darkpowderblue)}\rule{\linewidth}{.2mm}}
%%%%%%%%%%%%%%%%%%%%%%%%%%%%%%%%%%%%%%%%%%%%%%%%%%%%%%%%%%%%%%5%
\begin{example}
Find the area under one are of cycloid \\
$x=r(\theta-\sin\theta)~,~y=r(1-\cos\theta)$\\
\underline{\color{smalt(darkpowderblue)}Solution}\\
\begin{minipage}{0.6\textwidth}
$$A=\int_0^{2\pi}g(\theta)f'(\theta)d\theta=\int_0^{2\pi}r^2(1-\cos\theta)^2d\theta$$
$$=r^2\int_0^{2\theta}1-2\cos\theta+\cos^2\theta d\theta$$
$$=r^2\int_0^{2\theta}1-2\cos\theta+\cfrac{1}{2}+\cfrac{1}{2}cos2\theta d\theta$$
$$r^2[\theta-2\sin\theta+\cfrac{\theta}{2}+\cfrac{1}{4}\sin2\theta]|_0^{2\pi}=\boxed{3\pi r^2}$$
\end{minipage}
\begin{minipage}{0.6\textwidth}
\includegraphics[width=5cm]{2a.jpg}\\
$\boxed{f'(\theta)=r(1-\cos\theta)}$
\end{minipage}
\end{example}
%%%%%%%%%%%%%%%%%%%%%%%%%%%%%%%%%%%%%%%%%%%%%%%%%%%%%%%%%%%%%%5%%
\begin{exercise}
Q32~page~637 Find the area enclosed by the curve $x=t^2-2t~,~y=\sqrt{t}$ and the $y-axis$\\
\underline{\color{smalt(darkpowderblue)}Solution} \\
\begin{minipage}{0.34\textwidth}
$$A=\int_0^2 f(t)g'(t)dt$$
$$=\int_0^2 (t^2-2t)\cfrac{1}{2}t^{-\cfrac{1}{2}}dt$$
$$\vdots~~\vdots$$
\end{minipage}
\begin{minipage}{0.34\textwidth}
\includegraphics[width=4cm]{3a.jpg}
\end{minipage}
\begin{minipage}{0.34\textwidth}
$$Let~x=0\Rightarrow$$
$$t^2-2t=0$$
$$t=0~,~t=2$$
\end{minipage}
\end{exercise}
%%%%%%%%%%%%%%%%%%%%%%%%%%%%%%%%%%%%%%%%%%%%%%%%%%%%%%%%%%%%%%%%5%%
\textbf{\color{smalt(darkpowderblue)}\large \underline{Arc Length:}}
Let $x=f(t)~,~y=g(t)$
$$L=\int_a^b\sqrt{f'^2(t)+g'^2(t)}dt$$
\begin{remark}
the curve should be traversed once in $a\leq t\leq b$
\end{remark}
\noindent{\color{smalt(darkpowderblue)}\rule{\linewidth}{.2mm}}
%%%%%%%%%%%%%%%%%%%%%%%%%%%%%%%%%%%%%%%%%%%%%%%%%%%%%%%%%%%%%%%%%%
\begin{example}
Find the length of $x=\cos\theta~,~y=\sin\theta~,~0\leq\theta\leq 2\pi$\\
\underline{\color{smalt(darkpowderblue)}Solution} \\
\begin{minipage}{0.6\textwidth}
$$L=\int_0^{2\pi}\sqrt{\cos^2\theta+\sin^2\theta}d\theta$$
$$\int_0^{2\pi}1=2\pi$$
\end{minipage}
\begin{minipage}{0.6\textwidth}
\includegraphics[width=4cm]{4a.jpg}
\end{minipage}
\end{example}
%--------------------------------Theorem-----------------------------
\noindent{\color{smalt(darkpowderblue)}\rule{\linewidth}{.2mm}}
\begin{example}
Find the length of one are of the cycloid\\ $x=r(\theta-\sin\theta)~,~y=r(1-\cos\theta)$.\\
\underline{\color{smalt(darkpowderblue)}Solution} \\
$$L=\int_0^{2\pi}\sqrt{(\cfrac{dx}{d\theta})^2+(\cfrac{dy}{d\theta})^2}d\theta$$
$$=\int_0^{2\pi}\sqrt{r^2(1-\cos\theta)^2+r^2(\sin\theta)^2}d\theta$$
$$=r\int_0^{2\pi}\sqrt{1-2\cos\theta+\cos^2\theta+\sin^2\theta}d\theta$$
\begin{minipage}{0.5\textwidth}
$$=r\int_0^{2\pi}\sqrt{2-2cos\theta}d\theta$$
$$=r\int_0^{2\pi}\sqrt{4\sin^2\cfrac{\theta}{2}}d\theta$$
$$=2r\int_0^{2\pi}|\sin\cfrac{\theta}{2}|d\theta$$
$$=2r\int_0^{2\pi}\sin\cfrac{\theta}{2}$$
\end{minipage}
\begin{minipage}{0.5\textwidth}
$$2-2\cos\theta$$
$$=2(1-\cos\theta)$$
$$=4(\cfrac{1}{2}-\cfrac{1}{2}\cos\theta)$$
$$=4(\sin^2\cfrac{1}{2}\theta)$$
\end{minipage}
\begin{minipage}{0.6\textwidth}
$$2r(2)\cos\cfrac{\theta}{2}|_{2\pi}^0$$
$$4r(1+1)=8r$$
\end{minipage}
\begin{minipage}{0.5\textwidth}
\includegraphics[width=5cm]{2a.jpg}
\end{minipage}
\end{example}
\noindent{\color{smalt(darkpowderblue)}\rule{\linewidth}{.2mm}}
%%%%%%%%%%%%%%%%%%%%%%%%%%%%%%%%%%%%%%%%%%%%%%%%%%%%%%%%%%%%%%%
\textbf{\color{smalt(darkpowderblue)}\large \underline{Surfaces Area:}}
\begin{center}
    $S=2\pi\int_a^b g(t)\sqrt{f'^2(t)+g'^2(t)}dt~about~the~x-axis$\\
$about~the~x-axis~=2\pi\int_a^b ydL$\\
$S=2\pi\int_a^b f(t)\sqrt{f'^2(t)+g'^2(t)}dt~about~the~y-axis$
\end{center}
%%%%%%%%%%%%%%%%%%%%%%%%%%%%%%%%%%%%%%%%%%%%%%%%%%%%%%%%%%%%%%%%%5
\begin{example}
Find the area of the surface obtained by revolving one arc of the cycloid $x=r(\theta-\sin\theta)~,~y=r(1-\cos\theta)$ about $x-axis$\\
\underline{\color{smalt(darkpowderblue)}Solution:}\\
\begin{minipage}{0.6\textwidth}
\begin{center}
$S=2\pi\int_0^{2\pi}r(1-\cos\theta\sqrt{2(1-\cos\theta)}d\theta$\\
$\vdots$    
\end{center}
\end{minipage}
\begin{minipage}{0.6\textwidth}
\includegraphics[width=5cm]{5a.jpg}
\end{minipage}
\end{example}

%%%%%%%%%%%%%%%%%%%%%%%%%%%%%%%%%%%%%%%%%%%%%%%%%%%%%%%%%%%%%%%%%%%
\begin{example}
Find the area of a sphere of radius $r$ $\boxed{S=4\pi r^2}$\\
\underline{\color{smalt(darkpowderblue)}Solution} \\
\begin{minipage}{0.6\textwidth}
$$S=2\pi\int_0^\pi r\sin\theta\sqrt{(\cfrac{dx}{d\theta})^2+(\cfrac{dy}{d\theta})^2}d\theta$$
$$=2\pi r\int_0^\pi\sin\theta\sqrt{r^2}d\theta$$
$$=2\pi r^2\int_0^\pi\sin\theta d\theta$$
$$=2\pi r^2\cos\theta|_\pi^0$$
$$=2\pi r^2(1+1)=4\pi r^2$$
\end{minipage}
\begin{minipage}{0.6\textwidth}
\includegraphics[width=4cm]{5a.jpg}\\
$x=r\cos\theta$\\
$y=r\sin\theta$\\
$0\leq\theta\leq\pi$
\end{minipage}
\end{example}
\noindent{\color{smalt(darkpowderblue)}\rule{\linewidth}{.2mm}}
%%%%%%%%%%%%%%%%%%%%%%%%%%%%%%%%%%%%%%%%%%%%%%%%%%%%%%%%%%%%%%%%5%
\begin{problem}
 1, 3, 5, 7, 11, 12, 13, 15, 17, 18, 19, 28, 29, 30, 33, 34, 37, 39, 40, 41, 43, 57, 59, 60, 65, 69.
\end{problem}
%%%%%%%%%%%%%%%%%%%%%%%%%%%%%%%%%%%%%%%%%%%%%%%%%%%%%%%%%%%%%55%%
\section{Polar Coordinates.}
\animategraphics[height = 2.8in , controls]{1}{ani_}{0}{9}
\animategraphics[height = 2.8in , controls]{1}{anim_}{0}{9}\\
.\hspace{2cm}{\Large$r=\sin(a\theta)\hspace{5cm}r=\cos(a\theta)$}\\ \\
\animategraphics[height = 2.8in , controls]{1}{anim1_}{0}{4}
\animategraphics[height = 2.8in , controls]{1}{anima_}{0}{5}\\
.\hspace{2cm}{\Large$r=a+b\cos\theta\hspace{5cm}r=a+b\sin\theta$}\\ \\
\noindent{\color{smalt(darkpowderblue)}\rule{\linewidth}{.2mm}}
\begin{example}
Locate the following polar points .
\begin{enumerate}
    \item $(3,\cfrac{\pi}{4}) $
    \item $(-3,\cfrac{\pi}{4})$
    \item $(1,\pi)$
    \item $(-1,0)$
    \item $(1,-\pi)$
\end{enumerate}
\begin{minipage}{0.3\textwidth}
\includegraphics[width=5cm]{a1.jpg}
\end{minipage}
\begin{minipage}{.3\textwidth}
\includegraphics[width=5cm]{a2.jpg}
\end{minipage}
\hfill
\begin{minipage}{.3\textwidth}
\includegraphics[width=5cm]{a3.jpg}
\end{minipage}
\begin{minipage}{0.5\textwidth}
\underline{\textbf{\large}\color{smalt(darkpowderblue)}Solution} \\
\end{minipage}
\begin{minipage}{.5\textwidth}
\includegraphics[width=5cm]{a4.PNG}
\end{minipage} \\
\end{example}
%%%%%%%%%%%%%%%%%%%%%%%%%%%%%%%%%%%%%%%%%%%%%%%%%%%%%%%%%%%%%%%
\noindent{\color{smalt(darkpowderblue)}\rule{\linewidth}{.2mm}}
$x=rcos\theta \hspace{2cm} r^2=x^2+y^2\\y=rsin\theta \hspace{2cm} \theta=tan^{-\cfrac{y}{x}}$ \\
\begin{example}
Find the cartesion coordinate for the following polar points.
$1) ~(1, \cfrac{2\pi}{3}) \hspace{2cm} \\x=1*cos\cfrac{2\pi}{3}=-\cfrac{1}{2} \\y=1*sin\cfrac{2\pi}{3}=\cfrac{\sqrt{3}}{2}$\\
\end{example}
%%%%%%%%%%%%%%%%%%%%%%%%%%%%%%%%%%%%%%%%%%%%%%%%%%%%%%%%%%%%%%%%
\noindent{\color{smalt(darkpowderblue)}\rule{\linewidth}{.2mm}}
\begin{example}
Converte from cartesion to polar \\
\underline{\textbf{\large}\color{smalt(darkpowderblue)}Solution}
$(1,-\sqrt{3}) $
$r^2=1+3 =4 \Rightarrow r=2$,\\ \vspace{.5cm} $\theta=tan^{-1}\sqrt{3}/1 \Rightarrow \theta=300$\\
\end{example}
\noindent{\color{smalt(darkpowderblue)}\rule{\linewidth}{.2mm}}
%%%%%%%%%%%%%%%%%%%%%%%%%%%%%%%%%%%%%%%%%%%%%%%%%%%%%%%%%%%%%%%%
\large\textbf{{{{Polar Curves}}}} \\
\begin{example}
Sketch the following polar curves .
\begin{itemize}
    \item $ r=2$
    \begin{minipage}{0.5\textwidth}  
    $r^2=4 \Rightarrow x^2+y^2=4$
\end{minipage}
\begin{minipage}{0.5\textwidth}
\includegraphics[width=4cm]{a5.jpg}
\end{minipage}
%%%%%%%%%%%%%%%%%%%%%%%%%%%%%%%%%%%%%%%%%%%%%%%%%%%%%%%%%%%%%%%%%%%%%%%
    \item $ \theta=\cfrac{\pi}{4}$\\
    \begin{minipage}{0.6\textwidth}
      $tan\theta=\cfrac{y}{x}$\\
      $1=\cfrac{y}{x} \Rightarrow y=x$
\end{minipage}
 \begin{minipage}{0.5\textwidth}
\includegraphics[width=4cm]{a6.jpg}
\end{minipage}
%%%%%%%%%%%%%%%%%%%%%%%%%%%%%%%%%%%%%%%%%%%%%%%%%%%%%%%%%%%%%%%%%%%%%%%
\item  $r=2*cos\theta$\\
\begin{minipage}{0.6\textwidth}
$r^2=2*r*cos\theta \Rightarrow x^2+y^2=2x$ \\$x^2-2x+...\textbf{+1}+y^2=0 ...\textbf{+1}$ \\$(x=1)^2+y^2=1$
\end{minipage}
\begin{minipage}{0.5\textwidth}
\includegraphics[width=4cm]{a7.jpg}
\end{minipage}
%%%%%%%%%%%%%%%%%%%%%%%%%%%%%%%%%%%%%%%%%%%%%%%%%%%%%%%%%%%%%%%%%%%%%%%
\item 
\begin{minipage}{0.6\textwidth}
$r=-4*cos\theta$
\end{minipage}
\begin{minipage}{0.5\textwidth}
\includegraphics[width=4cm]{a8.jpg}\\
\end{minipage}
%%%%%%%%%%%%%%%%%%%%%%%%%%%%%%%%%%%%%%%%%%%%%%%%%%%%%%%%%%%%%%%%%%%%%%%
\item 
\begin{minipage}{0.6\textwidth}
$r=2*sin\theta$
\end{minipage}
\begin{minipage}{0.5\textwidth}
\includegraphics[width=4cm]{a9.jpg}\\
\end{minipage}
\item $r=-4*sin\theta$ (H.w)
\item $r=2*sin\theta-4*cos\theta$(H.W)
\item $r=3*sec\theta$\\ $r=\cfrac{3}{cos\theta}$ $\Rightarrow r*cos\theta =3\\ \Rightarrow x=3$
%%%%%%%%%%%%%%%%%%%%%%%%%%%%%%%%%%%%%%%%%%%%%%%%%%%%%%%%%%%%%%%%%%%%%%%
\item 
\begin{minipage}{0.6\textwidth}
$r=2*sin\theta$
\end{minipage}
\begin{minipage}{0.5\textwidth}
\includegraphics[width=3.5cm]{a10.jpg}\\
\end{minipage}
\item $r=-3*\csc\theta$ (H.W)
%%%%%%%%%%%%%%%%%%%%%%%%%%%%%%%%%%%%%%%%%%%%%%%%%%%%%%%%%%%%%%%%%%%%%%%
\item $r=1+sin\theta$\\First sketch the equation in the cartesion system \\
\begin{minipage}{0.32\textwidth}
	\begin{tabular}{ |c | c |  }
	\hline
	$\theta$ & $r$ \\
{\color{red}$0 \rightarrow 2$} & {\color{red} $1 \rightarrow 2$} \\
{\color{red} $\cfrac{\pi}{2} \rightarrow \pi $} & {\color{red}$2 \rightarrow 1$} \\
{\color{red}$\pi \rightarrow \cfrac{2\pi}{3}$} & {\color{red}$1 \rightarrow 0$}\\
{\color{red}$\cfrac{2\pi}{3}\rightarrow \pi $ } & {\color{red}$0 \rightarrow 1$}
	\end{tabular}
\end{minipage}
\begin{minipage}{0.32\textwidth}
\includegraphics[width=3.5cm]{a11.jpg}
\end{minipage}
\begin{minipage}{0.32\textwidth}
\includegraphics[width=3.5cm]{a12.jpg}
\end{minipage}
%%%%%%%%%%%%%%%%%%%%%%%%%%%%%%%%%%%%%%%%%%%%%%%%%%%%%%%%%%%%%%%%%%%%%%%
\item $r=1-sin\theta$ (H.W)
\item $r=1+cos\theta$ (H.W)
\item $r=1-cos\theta$ (H.W)
\item \begin{minipage}{0.6\textwidth}
$r=2+2cos\theta$ (H.W)
\end{minipage}
%%%%%%%%%%%%%%%%%%%%%%%%%%%%%%%%%%%%%%%%%%%%%%%%%%%%%%%%%%%%%%%%%%%%%%%
\begin{minipage}{0.5\textwidth}
\includegraphics[width=4cm]{a13.jpg}\\
\end{minipage}
\item $r=2+sin\theta$ (H.W)\\
\begin{minipage}{0.6\textwidth}
	\begin{tabular}{ |c | c |  }
	\hline
	$\theta$ & $r$ \\
{\color{red}$0 \rightarrow 2$} & {\color{red} $2 \rightarrow 3$} \\
{\color{red} $\cfrac{\pi}{2} \rightarrow \pi $} & {\color{red}$3 \rightarrow 2$} \\
{\color{red}$\pi \rightarrow \cfrac{2\pi}{3}$} & {\color{red}$2 \rightarrow 1$}\\
{\color{red}$\cfrac{2\pi}{3}\rightarrow \pi $ } & {\color{red}$1 \rightarrow 2$}
	\end{tabular}
\end{minipage}
\begin{minipage}{0.5\textwidth}
\includegraphics[width=4cm]{a14.jpg}\\
\end{minipage}
\item $r=3+2cos\theta$ (H.W)
\item $r=3-2cos\theta$ (H.W)
\end{itemize}
\end{example}
\noindent{\color{smalt(darkpowderblue)}\rule{\linewidth}{.2mm}}
\begin{example}
Sketch $r=1+2\cos\theta$\\
\underline{\textbf{\large}\color{smalt(darkpowderblue)}Solution} \\
\begin{minipage}{0.5\textwidth}
\includegraphics[width=5cm]{1b.jpg}
\end{minipage}
\begin{minipage}{0.45\textwidth}
$$r=0\Rightarrow$$
$$1+2\cos\theta=0$$
$$\cos\theta=-\cfrac{1}{2}$$
$$\theta=\cfrac{2}{3}\pi~,~\cfrac{4}{3}\pi$$
\end{minipage}\\
\begin{minipage}{0.5\textwidth}
	\begin{tabular}{ |c | c | c | }
	\hline
${\color{red}\theta}$ & ${\color{red}r}$\\
${\color{smalt(darkpowderblue)}0\rightarrow\cfrac{\pi}{2}}$ & ${\color{smalt(darkpowderblue)}3\rightarrow 1}$\\
${\color{smalt(darkpowderblue)}\cfrac{\pi}{2}\rightarrow\cfrac{\pi}{3}}$ & ${\color{smalt(darkpowderblue)}1\rightarrow0}$\\
${\color{smalt(darkpowderblue)}\cfrac{2}{3}\pi\rightarrow \pi}$ & ${\color{smalt(darkpowderblue)}0\rightarrow -1}$\\
${\color{smalt(darkpowderblue)}\pi\rightarrow\cfrac{4}{3}\pi}$& ${\color{smalt(darkpowderblue)}0\rightarrow-1}$\\
${\color{smalt(darkpowderblue)}\cfrac{4}{3}\pi\rightarrow\cfrac{3}{2}\pi}$& ${\color{smalt(darkpowderblue)}0\rightarrow1}$\\
${\color{smalt(darkpowderblue)}\cfrac{3}{2}\pi\rightarrow2\pi}$& ${\color{smalt(darkpowderblue)}1\rightarrow3}$\\
	\end{tabular}
\end{minipage}
\begin{minipage}{0.6\textwidth}
\includegraphics[width=5cm]{2b.jpg}
\end{minipage}
\end{example}

\noindent{\color{smalt(darkpowderblue)}\rule{\linewidth}{.2mm}}
$$\boxed{r=a+b\sin\theta}$$
%%%%%%%%%%%%%%%%%%%%%%%%%%%%%%%%%%%%%%%%%%%%%%%%%%%%%%%%%%%%%%
\begin{example}
Sketch $r=-3-6\sin\theta$\\
\underline{\textbf{\large}\color{smalt(darkpowderblue)}Solution} \\
\begin{center}
 \includegraphics[width=5cm]{3b.jpg}
\end{center}
\end{example}
\noindent{\color{smalt(darkpowderblue)}\rule{\linewidth}{.2mm}}
%%%%%%%%%%%%%%%%%%%%%%%%%%%%%%%%%%%%%%%%%%%%%%%%%%%%%%%%%%%%%%%
\begin{example}
Sketch $r=\cos2\theta$\\
\underline{\textbf{\large}\color{smalt(darkpowderblue)}Solution}\\
\begin{minipage}{0.5\textwidth}
	\begin{tabular}{ |c | c | c | }
	\hline
${\color{red}\theta}$ & ${\color{red}r}$\\
${\color{smalt(darkpowderblue)}0\rightarrow\cfrac{\pi}{4}}$ & ${\color{smalt(darkpowderblue)}1\rightarrow 0}$\\
${\color{smalt(darkpowderblue)}\cfrac{\pi}{4}\rightarrow\cfrac{\pi}{2}}$ & ${\color{smalt(darkpowderblue)}0\rightarrow-1}$\\
${\color{smalt(darkpowderblue)}\cfrac{\pi}{2}\rightarrow\cfrac{3}{4}\pi}$ & ${\color{smalt(darkpowderblue)}-1\rightarrow 0}$\\
${\color{smalt(darkpowderblue)}\cfrac{3}{4}\pi\rightarrow\pi}$& ${\color{smalt(darkpowderblue)}0\rightarrow1}$\\
${\color{smalt(darkpowderblue)}\pi\rightarrow\cfrac{5}{4}\pi}$& ${\color{smalt(darkpowderblue)}1\rightarrow0}$\\
${\color{smalt(darkpowderblue)}\cfrac{5}{4}\pi\rightarrow\cfrac{3}{2}\pi}$& ${\color{smalt(darkpowderblue)}0\rightarrow-1}$\\
${\color{smalt(darkpowderblue)}\cfrac{3}{2}\pi\rightarrow\cfrac{7}{4}\pi}$& ${\color{smalt(darkpowderblue)}-1\rightarrow0}$\\
${\color{smalt(darkpowderblue)}\cfrac{7}{4}\pi\rightarrow2\pi}$& ${\color{smalt(darkpowderblue)}0\rightarrow1}$\\
	\end{tabular}
\end{minipage}
\begin{minipage}{0.5\textwidth}
\includegraphics[width=6cm]{4b.jpg}
\end{minipage}
\begin{center}
 \includegraphics[width=8cm]{5b.jpg}    
\end{center}
\end{example}
%%%%%%%%%%%%%%%%%%%%%%%%%%%%%%%%%%%%%%%%%%%%%%%%%%%%%%%%%%%%%%%%%
\begin{exercise}
Sketch:\begin{enumerate}
    \item $r=\sin2\theta$
    \item $r=\cos3\theta$
    \item $r=\cos4\theta$
\end{enumerate}
\end{exercise}
%%%%%%%%%%%%%%%%%%%%%%%%%%%%%%%%%%%%%%%%%%%%%%%%%%%%%%%%%%%%%%%%
\begin{example}
Sketch $r^2=\cos2\theta$\\
\underline{\textbf{\large}\color{smalt(darkpowderblue)}Solution}
$$r=\sqrt{\cos2\theta}$$
$$r=-\sqrt{\cos2\theta}$$\\
\begin{minipage}{0.6\textwidth}
	\begin{tabular}{ |c | c | c | }
	\hline
${\color{red}\theta}$ & ${\color{red}r=\cos2\theta}$ &${\color{red}r^2=\cos2\theta}$\\
${\color{smalt(darkpowderblue)}0\rightarrow\cfrac{\pi}{4}}$ & ${\color{smalt(darkpowderblue)}1\rightarrow 0}$&
${\color{smalt(darkpowderblue)}1\rightarrow 0}$ \\
${\color{smalt(darkpowderblue)}\cfrac{\pi}{4}\rightarrow\cfrac{\pi}{2}}$ & ${\color{smalt(darkpowderblue)}0\rightarrow-1}$ &
${\color{smalt(darkpowderblue)}\times}$\\
${\color{smalt(darkpowderblue)}\cfrac{\pi}{2}\rightarrow\cfrac{3}{4}\pi}$ & ${\color{smalt(darkpowderblue)}-1\rightarrow 0}$& 
${\color{smalt(darkpowderblue)}\times}$\\
${\color{smalt(darkpowderblue)}\cfrac{3}{4}\pi\rightarrow\pi}$& ${\color{smalt(darkpowderblue)}0\rightarrow1}$& 
${\color{smalt(darkpowderblue)}0\rightarrow 1}$\\
${\color{smalt(darkpowderblue)}\pi\rightarrow\cfrac{5}{4}\pi}$& ${\color{smalt(darkpowderblue)}1\rightarrow0}$&
${\color{smalt(darkpowderblue)}1\rightarrow 0}$\\
${\color{smalt(darkpowderblue)}\cfrac{5}{4}\pi\rightarrow\cfrac{3}{2}\pi}$& ${\color{smalt(darkpowderblue)}0\rightarrow-1}$&
${\color{smalt(darkpowderblue)}\times}$\\
${\color{smalt(darkpowderblue)}\cfrac{3}{2}\pi\rightarrow\cfrac{7}{4}\pi}$& ${\color{smalt(darkpowderblue)}-1\rightarrow0}$& 
${\color{smalt(darkpowderblue)}\times}$\\
${\color{smalt(darkpowderblue)}\cfrac{7}{4}\pi\rightarrow2\pi}$& ${\color{smalt(darkpowderblue)}0\rightarrow1}$&
${\color{smalt(darkpowderblue)}0\rightarrow 1}$\\
	\end{tabular}
\end{minipage}
\begin{minipage}{0.6\textwidth}
\includegraphics[width=6cm]{6b.jpg}
\end{minipage}
\end{example}
%%%%%%%%%%%%%%%%%%%%%%%%%%%%%%%%%%%%%%%%%%%%%%%%%%%%%%%%%%%%%%%%%%
\noindent{\color{smalt(darkpowderblue)}\rule{\linewidth}{.2mm}}
{\Large\textbf{Symmetry:}}\\
When we sketch polar curves, it is sometimes helpful to take advantage of symmetry ,The following three rules are explained below : 
\begin{enumerate}
    \item If a polar equation is unchanged when $\theta$ is replaced by - $\theta$ , the curve is symmetric about the polar axis .
    \item If the equation is unchanged when r is replaced by -r , or when $\theta$ is replaced by $\theta+\pi$ the curve is symmetric about the pole.(This means that the 
    curve remains unchanged if we rotate it through $180^0$ about the origion)
    \item If the equation is unchanged when $\theta$ is replaced by $\pi-\theta$, the curve is symmetric about the vertical line $\theta=\cfrac{\pi}{2}$
\end{enumerate}
\begin{minipage}{0.3\textwidth}
\includegraphics[width=5cm]{f1.PNG}
\end{minipage}
\begin{minipage}{.3\textwidth}
\includegraphics[width=5cm]{f2.PNG}
\end{minipage}
\hfill
\begin{minipage}{.3\textwidth}
\includegraphics[width=5cm]{f3.PNG}
\end{minipage}
\noindent{\color{smalt(darkpowderblue)}\rule{\linewidth}{.2mm}}
%%%%%%%%%%%%%%%%%%%%%%%%%%%%%%%%%%%%%%%%%%%%%%%%%%%%%%%%%%%%%%%%%%
\begin{example}
$r=cos\theta$ is symmetric  
\begin{enumerate}
    \item about the polar axis .
    \item about the origin .
    \item about $\theta=\cfrac{\pi}{2}$
\end{enumerate}
\end{example}
\noindent{\color{smalt(darkpowderblue)}\rule{\linewidth}{.2mm}}
%%%%%%%%%%%%%%%%%%%%%%%%%%%%%%%%%%%%%%%%%%%%%%%%%%%%%%%%%%%%%%%%%%%%
Tangents in polar system $\cfrac{dy}{dx}$ \\
$x=rsoc\theta \hspace{1cm} y=rsin\theta$ \\
$\cfrac{dy}{dx}= \cfrac{\cfrac{dy}{d\theta}}{\cfrac{dx}{d\theta}}= \cfrac{\cfrac{dr}{d\theta}*sin\theta+rcos\theta}{\cfrac{dr}{d\theta}*cos\theta-r*sin\theta}$ \\
\noindent{\color{smalt(darkpowderblue)}\rule{\linewidth}{.2mm}}
%%%%%%%%%%%%%%%%%%%%%%%%%%%%%%%%%%%%%%%%%%%%%%%%%%%%%%%%%%%%%%%%%%%
\begin{example}
Let $r=1+sin\theta$
\begin{enumerate}
    \item Find the slope at $\theta=\cfrac{\pi}{3}$
    \item Find where the tangent is horizontal? Vertical ?
\end{enumerate}
\text{\color{smalt(darkpowderblue)}\underline{Solution}}: $x=rcos\theta \hspace{1cm}, y=rsin\theta $
\begin{enumerate}
    \item$ \cfrac{dy}{dx}=\cfrac{\cfrac{dr}{d\theta}*sin\theta+rcos\theta}{\cfrac{dr}{d\theta}*cos\theta-r*sin\theta} \\ \cfrac{dy}{d\theta}=cos\theta \\ 
    cos\cfrac{\pi}{3}=\cfrac{1}{2} \\ sin\cfrac{\pi}{3}=\cfrac{\sqrt{3}}{2} \\ \Rightarrow 
    \cfrac{dy}{dx}=\cfrac{cos\cfrac{\pi}{3}*sin\cfrac{\pi}{3}+cos\cfrac{\pi}{3}(1+sin\cfrac{\pi}{3})}{cos\cfrac{\pi}{3}*cos\cfrac{\pi}{3}+-(1+sin\cfrac{\pi}{3})*sin\cfrac{\pi}{3}} \\ =\cfrac{\cfrac{\sqrt{3}}{4} + \cfrac{1}{2}+\cfrac{\sqrt{3}}{4}}{\cfrac{1}{4}-\\\cfrac{\sqrt{3}}{2}-\cfrac{3}{4}}=-1 $ \\
Equation of the tangent: \\ Slope=-1 \\ $x_0=(1+sin\cfrac{\pi}{3}*cos\cfrac{\pi}{3}=(\cfrac{1}{2}+\cfrac{\sqrt{3}}{4}) )\\ 
y_0=(1+sin\cfrac{\pi}{3}*sin\cfrac{\pi}{3} )=(1+\cfrac{\sqrt{3}}{4})*\cfrac{\sqrt{3}}{2}=\cfrac{\sqrt{3}}{2}+\cfrac{3}{4}$\\
Equation : $y-\cfrac{\sqrt{3}}{2}+\cfrac{3}{4}=-1(x-(\cfrac{1}{2}+\cfrac{\sqrt{3}}{4}))$\\ 
\item $\cfrac{dy}{dx} =\cfrac{cos\theta*sin\theta+(1+sin\theta)*cos\theta}{cos\theta*cos\theta-(1+sin\theta)*sin\theta} \\ 
\cfrac{2*cos\theta*sin\theta+cos\theta}{cos^2\theta-sin^2\theta -sin\theta} = \cfrac{sin2\theta+cos\theta}{cos2\theta-sin\theta} \\ \cfrac{dy}{dx}=0 \Rightarrow 
2*cos\theta*sin\theta+cos\theta=0 \Rightarrow cos\theta(2sin\theta+1)=0 \\ \color{red}cos\theta=0 \hspace{1cm}$ or $sin\theta=-\cfrac{1}{2} \\ 
\theta=\cfrac{\pi}{2},\cfrac{3\pi}{2} \hspace{1cm} \theta=\cfrac{7\\pi}{6},\cfrac{11\pi}{6} \\
\Rightarrow \cfrac{dx}{d\theta}=0 \Rightarrow cos^2\theta-sin^2\theta-sin\theta=0 \\ 1-sin^2\theta-sin\theta
-sin\theta=0 \\ (2sin\theta-1)(sin\theta+1)=0 \\ \color{red} sin\theta=\cfrac{1}{2} \hspace{1cm} , sin\theta=-1 \\ \theta=\cfrac{\pi}{6} , \cfrac{5\pi}{6},\cfrac{3\pi}{2}$ 
\\ 
H.T $\theta=\cfrac{7\pi}{6}, \cfrac{11\pi}{6}, \cfrac{\pi}{2}$ \\ V.T $\theta=\cfrac{\pi}{6}, \cfrac{5\pi}{6} , \cfrac{3\pi}{2}$ \\ 
at $\theta=\cfrac{3\pi}{2} \\ \lim_{\theta \to \cfrac{3\pi }{2}+} \cfrac{dy}{dx}=\lim_{\theta \to \cfrac{3\pi}{2}}\cfrac{sin2\theta +cps\theta}{cos2\theta-sin\theta} $ 
\hspace{2cm} $(\cfrac{0}{0})$ \\ 
L'Hopital $=\lim_{\theta \to \cfrac{3\pi}{2}+} =\cfrac{2\cos2 \theta-\sin \theta}{2\sin 2\theta - \cos \theta}= -\infty$ 
\end{enumerate} 
\end{example}
\noindent{\color{smalt(darkpowderblue)}\rule{\linewidth}{.2mm}}
%%%%%%%%%%%%%%%%%%%%%%%%%%%%%%%%%%%%%%%%%%%%%%%%%%%%%%%%%%%%%%%%%%%
\begin{problem}
1, 3, 7, 8, 9, 11, 13, 15, 16, 17, 19, 21, 24, 25, 29, 30, 31, 34, 37, 39, 40, 42, 43, 47, 57, 58, 61, 63, 65, 67, 70.
\end{problem}
%%%%%%%%%%%%%%%%%%%%%%%%%%%%%%%%%%%%%%%%%%%%%%%%%%%%%%%%%%%%%%%%%%%
\section{Area and length}
\begin{minipage}{0.5\textwidth}
$$A=\cfrac{1}{2}\int_{a}^{b}r^2 d\theta$$
\end{minipage}
\begin{minipage}{0.5\textwidth}
\includegraphics[width=5cm]{f4.jpg}
\end{minipage}\\
%%%%%%%%%%%%%%%%%%%%%%%%%%%%%%%%%%%%%%%%%%%%%%%%%%%%%%%%%%%%%%%%%%%%
\begin{example}
Find the area of one leaf of the rose $r=\cos2\theta$\\
{\color{smalt(darkpowderblue)}\underline{Solution:}}
\begin{minipage}{0.5\textwidth}
  $$A=2*\cfrac{1}{2} *\int_{0}^{\cfrac{\pi}{4}} (\cos 2\theta)^2 d\theta$$
  $$A= \int_{0}^{\cfrac{\pi}{2}}(\cfrac{1}{2}+\cfrac{1}{2}*\cos 4\theta )d\theta$$
  $$A=\cfrac{\theta}{2}+\cfrac{1}{8}\sin 4\theta |_{0}^{\cfrac{\pi}{2}}$$
  $$A=\cfrac{\pi}{4}$$ 
\end{minipage}
\begin{minipage}{0.4\textwidth}
\includegraphics[width=5cm]{f5.jpg}
\end{minipage}
\end{example}
\noindent{\color{smalt(darkpowderblue)}\rule{\linewidth}{.2mm}}
\begin{example}
Find the area of the region that lies \underline{inside} $r=3\sin\theta$ and \underline{outside} $r=1+\sin\theta.$\\
{\color{smalt(darkpowderblue)} \underline{Solution:}} 
\begin{minipage}{0.5\textwidth}
$$r=f(\theta)$$
$$A=\cfrac{1}{2}\int_a^b r^2d\theta$$
\end{minipage}
\begin{minipage}{0.6\textwidth}
\includegraphics[width=4cm]{1c.jpg}
\end{minipage}\\
\begin{minipage}{0.5\textwidth}
\includegraphics[width=5cm]{2c.jpg}
\end{minipage}
\begin{minipage}{0.5\textwidth}
$$A=\cfrac{1}{2}\int_{\cfrac{\pi}{6}}^{\cfrac{5}{6}\pi}(3\sin\theta)^2-(1+\sin\theta)^2 d\theta$$
$$=2.\cfrac{1}{2}\int_{\cfrac{\pi}{6}}^{\cfrac{\pi}{2}}(3\sin\theta)^2-(1+\sin\theta)^2 d\theta$$
\end{minipage}
\end{example}
\noindent{\color{smalt(darkpowderblue)}\rule{\linewidth}{.2mm}}
%%%%%%%%%%%%%%%%%%%%%%%%%%%%%%%%%%%%%%%%%%%%%%%%%%%%%%%%%%%%%%%55%5
\begin{example}
Find the area inside both $r=1+\cos\theta~,~r=1-\cos\theta$\\
{\color{smalt(darkpowderblue)} \underline{Solution:}}
\begin{minipage}{0.4\textwidth}
\includegraphics[width=5cm]{3c.jpg}
\end{minipage}
\begin{minipage}{0.4\textwidth}
$$A=4.\cfrac{1}{2}\int_0^\frac{\pi}{2}(1-\cos\theta)^2 d\theta$$
$$or~A=4.\frac{1}{2}\int_\frac{\pi}{2}^\pi(1-\cos\theta)^2 d\theta$$
\end{minipage}
\end{example}
\noindent{\color{smalt(darkpowderblue)}\rule{\linewidth}{.2mm}}
%%%%%%%%%%%%%%%%%%%%%%%%%%%%%%%%%%%%%%%%%%%%%%%%%%%%%%%%%%%%%%%%
\begin{example}
Find the area inside the inner loop of $r=1+2\cos\theta$\\
{\color{smalt(darkpowderblue)} \underline{Solution:}} \\ 
\begin{minipage}{0.5\textwidth}
$$r=0$$
$$\Rightarrow 1+2\cos\theta=0$$
$$\Rightarrow\cos\theta=-\cfrac{1}{2}$$
$$\Rightarrow\theta=\cfrac{2}{3}\pi~,\cfrac{4}{3}\pi$$
$$A=2.\cfrac{1}{2}\int_{\cfrac{2}{3}\pi}^\pi(1+2\cos\theta)^2 d\theta$$
$$=2.\cfrac{1}{2}\int_{\pi}^{\cfrac{4}{3}\pi}(1+2\cos\theta)^2 d\theta$$
$$=\cfrac{1}{2}\int_{\cfrac{2}{3}\pi}^{\cfrac{4}{3}\pi}(1+2\cos\theta)^2 d\theta$$
\end{minipage}
\begin{minipage}{0.5\textwidth}
\includegraphics[width=6cm]{4c.jpg}
\end{minipage}
\end{example}
%%%%%%%%%%%%%%%%%%%%%%%%%%%%%%%%%%%%%%%%%%%%%%%%%%%%%%%%%%%%%%%%
\noindent{\color{smalt(darkpowderblue)}\rule{\linewidth}{.2mm}}
\begin{example}
Find the area the lies between the inner and the outer loop of $r=1+2\cos\theta$. \\
{\color{smalt(darkpowderblue)} \underline{Solution:}}\\ 
\begin{minipage}{0.75\textwidth}
$A=2.\cfrac{1}{2}\left(\int_0^{\cfrac{2}{3}\pi}(1+2\cos\theta)^2 d\theta-\int_{\cfrac{2}{3}\pi}^\pi(1+\cos\theta)^2d \theta\right)$
\end{minipage}
\begin{minipage}{0.5\textwidth}
\includegraphics[width=5cm]{4c.jpg}
\end{minipage}
\end{example}
\noindent{\color{smalt(darkpowderblue)}\rule{\linewidth}{.2mm}}
%%%%%%%%%%%%%%%%%%%%%%%%%%%%%%%%%%%%%%%%%%%%%%%%%%%%%%%%%%%%%%%%
\begin{example}
Find the area... inside both $r=\cos2\theta~,~r=\sin2\theta.$\\
{\color{smalt(darkpowderblue)} \underline{Solution:}}
\begin{minipage}{0.5\textwidth}
$$A=8.2.\cfrac{1}{2}\int_0^\frac{\pi}{8}(\sin2\theta)^2 d\theta$$
$$or~=16.\cfrac{1}{2}\int_\frac{\pi}{8}^\frac{\pi}{4}(\cos2\theta)^2 d\theta$$
\end{minipage}
\begin{minipage}{0.5\textwidth}
\includegraphics[width=6cm]{5c.jpg}
\end{minipage}
\end{example}
\noindent{\color{smalt(darkpowderblue)}\rule{\linewidth}{.2mm}}
%%%%%%%%%%%%%%%%%%%%%%%%%%%%%%%%%%%%%%%%%%%%%%%%%%%%%%%%%%%%%%%%%%%
\begin{problem}
2, 3, 5, 6, 7, 8, 9, 11, 13, 17, 21, 23, 24, 25, 26, 28, 29, 31, 33, 35, 37, 41, 45, 47.
\end{problem}
%%%%%%%%%%%%%%%%%%%%%%%%%%%%%%%%%%%%%%%%%%%%%%%%%%%%%%%%%%%%%%%%%%%%%%%%%%%%%%%%%%%%%%%%%%%%%%%%%%%%%%%%%%%%%%%%%%%%%%%%%%%%%%%%%%%%%%%%%%%%%%%%%%%%
\begin{example}
Find the area of the loop $r^2=9\cos 2\theta$ 
{\color{smalt(darkpowderblue)}\underline {Solution:}}\\
\begin{minipage}{0.4\textwidth}
$A=2*\frac{1}{2} \int_{0}^{\frac{\pi}{4}} 9\cos 2\theta d\theta$ \\
$A=\frac{9}{2} \sin 2\theta |_{0}^{\frac{\pi}{4}} \\A= \frac{9}{2}$ \\
\end{minipage}
\begin{minipage}{0.5\textwidth}
\includegraphics[width=5cm]{x3.jpg}
\end{minipage} \\
\noindent{\color{smalt(darkpowderblue)}\rule{\linewidth}{.2mm}}\\
{\color{smalt(darkpowderblue)}Parametric Equation}
\begin{itemize}
    \item $x=f(\theta) *cos(\theta)$ \hspace{1cm} $r=f(\theta)$
    \item $y=f(\theta) *sin(\theta)$
\end{itemize}
\end{example}
{\color{smalt(darkpowderblue)}\underline {Arc Length :}}\\
$$L={\Huge \int_{a}^{b}}\sqrt{(\cfrac{dx}{d\theta})^2+(\cfrac{dy}{d\theta})^2} d\theta $$
$$={\Huge \int_{a}^{b}\sqrt{(r'*cos(\theta)-r*sin(\theta))^2+(r'*sin(\theta)+r*cos(\theta))^2 }}d\theta$$\\
 $={\Huge \int_{a}^{b} \sqrt{ (r')^2*cos(\theta)^2 -2*r'*r*cos(\theta)*sin(\theta)+(r')^2*sin(\theta)^2+2*r'*r*cos(\theta)*sin(\theta)+r^2*cos^2(\theta)}}$\\
 $$L= {\Huge \int_{a}^{b} \sqrt{(r')^2+r^2}d\theta}$$
\begin{example}
Find the length of the cardiod $r=1+\sin{\theta}$
{\color{smalt(darkpowderblue)}\underline {Solution:}}\\
\begin{align*}
    & L= {\Huge \int_{0}^{2\pi} \sqrt{(r')^2+r^2}d\theta}\\
    & = {\Huge \int_{0}^{2\pi} \sqrt{(1+\sin{\theta})^2+\cos^2(\theta)}d\theta }\\
    &={\Huge \int_{0}^{2\pi} \sqrt{1+2\sin{\theta}+\sin^2{\theta}+\cos^2(\theta)}d\theta }\\
    &={\Huge \int_{0}^{2\pi} \sqrt{2+2\sin{\theta}}d\theta }*\color{red}\sqrt{\cfrac{2-2\sin{\theta}}{2-2\sin{\theta}}}\\
     &={\Huge \int_{0}^{2\pi}} \sqrt{\cfrac{4(1-\sin^2{\theta})}{2(1-\sin{\theta})}}\\
        &={\Huge \int_{0}^{2\pi}} {\cfrac{2\sqrt{\cos{^2{\theta}}}}{\sqrt{2(1-\sin{\theta})}}}\\
         &={\Huge \int_{0}^{2\pi}} {\cfrac{2\left|  {\cos{{\theta}}}\right|}{\sqrt{2(1-\sin{\theta})}}}\\
\end{align*}
\end{example}
\noindent{\color{smalt(darkpowderblue)}\rule{\linewidth}{.2mm}}
\begin{example}
\begin{enumerate}
    \item $\delta={\Huge \int_{a}^{b}} {2\pi r*\sin{\theta}*\sqrt{(r')^2+r^2}d\theta}$ about the polar axis.
    \item Find the surface area generated by rotating the lemniscate\\ $r^2=\cos{2\theta}$ about the polar axis .
\end{enumerate}
{\color{smalt(darkpowderblue)}\underline {Solution:}}\\
$\delta=2{\Huge \int_{0}^{\frac{\pi}{4}}} {2\pi r*\sin{\theta}*\sqrt{(r')^2+r^2}d\theta}\\
={\Huge 4\pi \int_{0}^{\frac{\pi}{4}}} { r*\sin{\theta}*\sqrt{(\cos{2\theta})+\cfrac{sin^2(2\theta)}{cos(2\theta)}}d\theta}\\
={\Huge 4\pi \int_{0}^{\frac{\pi}{4}}} { r*\sin{\theta}*\sqrt{\cfrac{1}{cos(2\theta)}}d\theta}\\
={\Huge 4\pi \int_{0}^{\frac{\pi}{4}}} { r*\sin{\theta}*\cfrac{1}{r}d\theta}\\
=4\pi *\cos{\theta} *(0-\cfrac{\pi}{4})\\
=4\pi *(1-\cfrac{1}{\sqrt{2}})$
\end{example}
\setcounter{chapter}{11}
\chapter{Vectors and Geometry of Space}
\section{3 Dimensional coordinate system}
\includegraphics[width=5cm]{calc3.png}\\
\begin{itemize}
    \item $XZ-plane$
    \item $XY-plane$
    \item $YZ-plane$
\end{itemize}
\includegraphics[width=5cm]{calc31.png}\\
\begin{example}
Describe the following equations :
\begin{enumerate}
\item
\begin{enumerate}
    \item  $y=3$ in $2D(R^2)$
    \item $y=3$ in $3D(R^3)$ (Plane parallel $xz-plane$)
\end{enumerate}
    \item $z^2=1 $ in $R^3$ \begin{itemize}
        \item z=1 plane above parallel xy-plane
                \item z=-1 plane below parallel xy-plane
    \end{itemize}
\item $x^2+y^2=1 ~ \& z=2$
\item $y=x$ in $R^3$
\begin{itemize}
    \item plane $\cfrac{\pi}{4}$ with XZ-plane , YZ-plane
\end{itemize}
\end{enumerate}
{\color{smalt(darkpowderblue)}\underline {Solution:}}\\
\begin{enumerate}
    \item 
    \begin{enumerate}
     \item \includegraphics[width=5cm]{calc33.png}\\
    \item \includegraphics[width=5cm]{calc32.png}\\
        \item \includegraphics[width=5cm]{calc34.png}\\
\end{enumerate}
           \item \includegraphics[width=5cm]{calc34.png}\\
    \item \includegraphics[width=5cm]{calc35.png}\\
    \item \includegraphics[width=5cm]{calc36.png}\\
\end{enumerate}
\end{example}
%%%%%%%%%%%%%%%%%%%%%%%%%%%%%%%%%%%%%%%%%%%%%%%%%%%%%%%%%%%%%%%%%%%%%%%%%%%%%%%%%%%%%%%%%%%%%%%%%%%%%%%%%%%%%%%%%%%
\section*{3D Coordinate System}
\begin{center}
    \includegraphics[width=5cm]{1d.png}\\
{\large\color{smalt(darkpowderblue)} $R^3~,V_3$}
\end{center}
\begin{itemize}
    \item $(x_1,y_1,z_1)~,~(x_2,y_2~,~z_2)$, Distance:\\
    $D=\sqrt{(x_2-x_1)^2+(y_2-y_1)^2+(z_2-z_1)^2}$
    \item $(h,k,l)~,r$
    $$(x-h)^2+(y-k)^2+(z-l)^2=r^2$$
    $$Sphere~center~(h,k,l)$$
    $$radius~r$$
\end{itemize}
%%%%%%%%%%%%%%%%%%%%%%%%%%%%%%%%%%%%%%%%%%%%%%%%%%%%%%%%%%%%%%%%%%%%%%%%%%%%%
\noindent{\color{smalt(darkpowderblue)}\rule{\linewidth}{.2mm}}
\begin{example}
Describe the region represented by: $1\leq x^2+y^2+z^2\leq 4$\\
{\color{smalt(darkpowderblue)}\underline{Solution}}\\
The equation represented the region between:\\
The sphere of center (0,0,0), radios 2\\
and the sphere of center (0,0,0), radios 1
\end{example}
%%%%%%%%%%%%%%%%%%%%%%%%%%%%%%%%%%%%%%%%%%%%%%%%%%%%%%%%%%%%%%%%%%%%%%%%%%%%%
\begin{example}
Find an equation of the largest sphere of center (5,4,9) in the first octant.\\
{\color{smalt(darkpowderblue)}\underline{Solution}}\\
\begin{minipage}{0.5\textwidth}
$$r=4$$
$$(x-5)^2+(y-4)^2+(z-9)^2=16$$
\end{minipage}
\begin{minipage}{0.5\textwidth}
\includegraphics[width=5cm]{2d.png}
\end{minipage}
\end{example}
%%%%%%%%%%%%%%%%%%%%%%%%%%%%%%%%%%%%%%%%%%%%%%%%%%%%%%%%%%%%%%%%%%%%%%%%%%%%
\begin{example}
Find the distance between the point (2,3,-4) and :\\
\begin{minipage}{0.5\textwidth}
\begin{enumerate}
    \item the $x-axis$
    $$d=\sqrt{3^2+4^2}=5$$
    \item the $xz-plane$
    $$d=3$$
\end{enumerate}
\end{minipage}
\begin{minipage}{0.5\textwidth}
\includegraphics[width=5cm]{3d.png}
\end{minipage}
\end{example}
\noindent{\color{smalt(darkpowderblue)}\rule{\linewidth}{.2mm}}
%%%%%%%%%%%%%%%%%%%%%%%%%%%%%%%%%%%%%%%%%%%%%%%%%%%%%%%%%%%%%%%%%%%%%
%%%%%5
\begin{problem}
2,3,7,9,10,11,13,14,15,17,19,20,21,23,25,29,32,33,35,38
\end{problem}
\section{Vectors}
%%%%%%%%%%%%%%%%%%%%%%%%%%%%%%%%%%%%%%%%%%%%%%%%%%%%%%%%%%%%%%%%%%%%%%%%%%
\begin{minipage}{0.5\textwidth}
\includegraphics[width=5cm]{4d.jpg}
\end{minipage}
\begin{minipage}{0.5\textwidth}
$$\overrightarrow{v}=\overrightarrow{PQ}=<x_2-x_1,y_2-y_1>$$
\end{minipage}
in $2D~\overrightarrow{a}=<x,y>$\hfill $|\overrightarrow{a}|=\sqrt{x^2+y^2}$\\
in $3D~\overrightarrow{a}=<x,y,z>$\hfill $|\overrightarrow{a}|=\sqrt{x^2+y^2+z^2}$\\
if $\overrightarrow{a}=<a_1,a_2,a_3,~\overrightarrow{b}=<b_1,b_2,b_3>\Rightarrow$\\
$$\overrightarrow{a}+\overrightarrow{b}=<a_1+b_1,a_2+b_2,a_3+b_3>$$
$$c\overrightarrow{a}=<ca_1,ca_2,ca_3>$$
$\overrightarrow{a}=<a_1,a_2,a_3>=a_1\underbrace{<1,0,0>}+a_2\underbrace{<0,1,0>}+a_3\underbrace{<0,0,1>}$\\
\hspace*{5.5cm} {\color{red}$i\hspace{3cm}j\hspace{3cm}k$}
$$=a_1 i+a_2 j+a_3k$$
%%%%%%%%%%%%%%%%%%%%%%%%%%%%%%%%%%%%%%%%%%%%%%%%%%%%%%%%%%%%%%%%%%%%%
\begin{definition}[Unit vector:]
 a vector $\overrightarrow{u}$ is called a unit if $|\overrightarrow{u}|=1$
\end{definition}
%%%%%%%%%%%%%%%%%%%%%%%%%%%%%%%%%%%%%%%%%%%%%%%%%%%%%%%%%%%%%%%%%%%%%%%%
\noindent{\color{smalt(darkpowderblue)}\rule{\linewidth}{.2mm}}
\begin{example}
Find a unit vector in the opposite direction of $\overrightarrow{v}=<2,-2,1>.$\\
{\color{smalt(darkpowderblue)}\underline{Solution:}} 
$|\overrightarrow{u}|=\cfrac{\overrightarrow{v}}{|\overrightarrow{v}|}=\cfrac{-1}{\sqrt{4+4+1}}<2,-2,1>=<\cfrac{-2}{3},\cfrac{2}{3},\cfrac{-1}{3}>$
\end{example}
\noindent{\color{smalt(darkpowderblue)}\rule{\linewidth}{.2mm}}
\begin{problem}
7,11,13,15,17,19,21,23,24,25,37,41,42
\end{problem}
%%%%%%%%%%%%%%%%%%%%%%%%%%%%%%%%%%%%%%%%%%%%%%%%%%%%%%%%%%%%%%%%%%%%%%
\section{The Dot Product}
\begin{definition}
if $\overrightarrow{a}=<a_1,a_2,a_3>~,~\overrightarrow{b}=<b_1,b_2,b_3>\Rightarrow$
$$\overrightarrow{a}.\overrightarrow{b}=a_1b_1+a_2b_2+a_3b_3$$
\end{definition} 
%%%%%%%%%%%%%%%%%%%%%%%%%%%%%%%%%%%%%%%%%%%%%%%%%%%%%%%%%%%%%%%%%%%%%%%%
\begin{example}
if $\overrightarrow{a}=<2,1,-2>~,~\overrightarrow{b}=<1,1,3>$\\
{\color{smalt(darkpowderblue)}\underline{Solution}}\\
$\overrightarrow{a}.\overrightarrow{b}=2+1-6=-3$\\
$\overrightarrow{a}.\overrightarrow{a}=a_1^2+a_2^2+a_3^2=(\sqrt{a_1^2+a_2^2+a_3^2})^2=|\overrightarrow{a}|^2$
\end{example}
\noindent{\color{smalt(darkpowderblue)}\rule{\linewidth}{.2mm}}
%%%%%%%%%%%%%%%%%%%%%%%%%%%%%%%%%%%%%%%%%%%%%%%%%%%%%%%%%%%%%%%%%%%%%%%%%
{\color{smalt(darkpowderblue)}Properties:}
\begin{enumerate}
    \item $\vv{a}.\vv{b}=\vv{b}.\vv{a}$
    \item $\vv{a}.\vv{a}=|\vv{a}|^2$
    \item $\vv{a}.(\vv{b}+\vv{c})=\vv{a}.\vv{b}+\vv{a}.\vv{c}$
    \item $(c\vv{a}).\vv{b}=\vv{a}.(c\vv{b})=c(\vv{a}.\vv{b})$
    \item $\vv{0}.\vv{a}=0$
\end{enumerate}
\begin{theorem}
if $\theta$ is the angle between $\vv{a}~\&~\vv{b}$ then $\vv{a}.\vv{b}=|\vv{a}||\vv{b}|\cos\theta\\
\theta\in[0,\pi]$
\end{theorem}
{\color{smalt(darkpowderblue)}Corollary 1:}
$\cos{\theta}=\cfrac{\vv{a}.\vv{b}}{|\vv{a}||\vv{b}|}$ $|\vv{a}|\neq0~,~|\vv{b}|\neq0$\\
\begin{example}
Find the angle between $\vv{a}=<2,2,-1>~,~\vv{b}=<5,-3,-2>$\\
$\cos{\theta}=\cfrac{\vv{a}.\vv{b}}{|\vv{a}||\vv{b}|}=\cfrac{10-6-2}{3\sqrt{25+9+4}}=\cfrac{2}{3\sqrt{38}}$\\
$g=cos^{-1}(\cfrac{2}{3\sqrt{38}}\approx 1.46(84)^\circ$
\end{example}
\noindent{\color{smalt(darkpowderblue)}\rule{\linewidth}{.2mm}}
{\color{smalt(darkpowderblue)}Corollary 2:} Two non-zero vectors are orthogonal iff $\vv{a}.\vv{b}=0$
\begin{example}
if $\vv{a}=<2,1,-2>~,~\vv{b}=<c,2,1>$\\
Find $c$ such that $\vv{a}\bot\vv{b}=0$\\
$\vv{a}.\vv{b}=0\Leftrightarrow 2c+2-2=0\Leftrightarrow c=0$
\end{example}
$\vv{a} =<a_1,a_2,a_3>\Rightarrow\\$
$ \vv{a} . \vv{b} = a_1b_1 + a_2b_2 + a_3b_3$ \\
$\vv{b} =<b_1 , b_2 , b_2>$ \hspace{1.1cm} $\vv{a} . \vv{b} = |\vv{a}| . |\vv{b}| \; \; cos\theta$
\hspace*{3.5cm} $\vv{a} . \vv{b} = 0 \leftrightarrow  \vv{a} \bot \vv{b}$
\begin{itemize}
\item  $-|a||b| \leq \vv{a} . \vv{b} \leq |a| |b|$\\
\hspace*{1.4cm} $| \vv{a} . \vv{b}| \leq |a| |b|$
\end{itemize}
\noindent{\color{smalt(darkpowderblue)}\rule{\linewidth}{.2mm}}
\begin{example}
if $|\vv{a}| = 2$ , $|\vv{b}| = 3$ , $\theta = \dfrac{2}{3} \pi$. Find $|\vv{a} - 2 \vv{b}|$. 
$|\vv{u}|^2 = \vv{u} . \vv{u}$ \hspace{3cm}
$|\vv{a} - 2 \vv{b}|^2$  = $(\vv{a} - 2 \vv{b}) . (\vv{a} -2 \vv{b})$\\
\hspace*{7.1cm}    = $\vv{a} . \vv{a} - 2 \vv{a}
\vv{b} - 2 \vv{b} . \vv{a} + 4 \vv{b} . \vv{b}$\\
\hspace*{7.1cm}           = $|\vv{a}|^2 - 4 |\vv{a}| |\vv{b}| cos\theta + 4|\vv{b}|^2$\\
\hspace*{7.1cm}            = $4 - 4 . 2 .3 \left(\dfrac{-1}{2}\right) + 4.9$\\
\hspace*{7.1cm}           = $16 + 36 = 52$ 
$|\vv{a} - 2\vv{b}| = \sqrt{52}$\\
$\boxed{|\vv{a} + \vv{b}| = \sqrt{|a|^2 + |b|^2 -2 \; |a| \; |b| \; cos\theta}}$
\end{example}
\noindent{\color{smalt(darkpowderblue)}\rule{\linewidth}{.2mm}}
\begin{example}
Prove that $|\vv{a}  + \vv{b}| \leq |\vv{a}| + |\vv{b}| $.\\
\hspace*{1cm} \emph{Pf:}\\
$|\vv{a} + \vv{b}|^2 = (\vv{a} + \vv{b}) . (\vv{a} . \vv{b})$\\
\hspace*{1.35cm} = $|\vv{a}|^2 + |\vv{b}|^2 + 2 \; \vv{a} . \vv{b} \leq |\vv{a}|^2 + |\vv{b}|^2 + 2 \; |a| \; |b|$\\
\hspace*{4.95cm} = $\underbrace{(|a| + |b| )^2}$\\
$\Rightarrow \; |\vv{a} + \vv{b}| \; \leq \; |\vv{a}| \; + \; |\vv{b}|$

\end{example}
\noindent{\color{smalt(darkpowderblue)}\rule{\linewidth}{.2mm}}
\begin{paracol}{2}
Direction Angles.\\
is the angle b/w $\vv{v}$ \& the  $x - axis$ 
$\beta$ \\
$\gamma$\\
\switchcolumn
Direction Cosines.
$cos \alpha = \dfrac{a_1}{|\vv{v}|}$
$cos \beta = \dfrac{a_2}{|\vv{v}|}$
$cos\gamma = \dfrac{a_3}{|\vv{v}|}$
\end{paracol}

$a_1 = cos \alpha |\vv{v}|$ , $a_2 = cos \beta |\vv{v}|$, $a_3 = cos \gamma |\vv{v}|$ \\
$\vv{v} = |\vv{v}| <cos\alpha,cos\beta,<\gamma> $\\
$\dfrac{\vv{v}}{|\vv{v}|} = <cos \alpha , cos \beta , cos \gamma>$ 

$\Rightarrow cos^2 \alpha + cos^2 \beta + cos^2 \gamma = `1$\\
\noindent{\color{smalt(darkpowderblue)}\rule{\linewidth}{.2mm}}
\begin{example}
can $\alpha = \cfrac{\pi}{4}$ , $\beta = \cfrac{\pi}{4}$ , $\gamma = \cfrac{\pi}{4}$ be direction cosines.
Answer. No, because \\
$cos^2 \cfrac{\pi}{4} + cos^2 \cfrac{\pi}{4} + cos^2 \cfrac{\pi}{4} = \cfrac{3}{2} \not= 1$
\end{example}
\begin{exercise}
if $\vv{a} = <1 , 2, 3>$ Find direction angles.
\end{exercise}
\large{Projections}
\begin{itemize}
\item Scalar Projection: \\
\begin{paracol}{3}
${comp}_{\vv{a}}^{\vv{b}} = \dfrac{\vv{a} . \vv{b}}{|\vv{a}|^2}$ 
\switchcolumn

$\dfrac{|\vv{a}| \; |\vv{b}| \; cos\theta}{|\vv{a}|} = \mathcal{L}$\\
$\dfrac{\vv{a} . \vv{b}}{\vv{a}}$ = 

\switchcolumn

$\vv{v} = \mathcal{L} \; \dfrac{\vv{a}}{|\vv{a}|}$ \\
\hspace*{0.5cm} $= \dfrac{\vv{a} . \vv{a}}{|\vv{a}|^2} \; \vv{a} $
\end{paracol}
\item Vector Projection:\\
${Proj}_{\vv{a}}^{\vv{b}} = \dfrac{\vv{a} . \vv{b}}{|\vv{a}|^2} \; \vv{a}$
\end{itemize}
\noindent{\color{smalt(darkpowderblue)}\rule{\linewidth}{.2mm}}
\begin{example}
if $\vv{a} = <2,1,-1>$  \\
\hspace*{5cm} Find\\
\hspace*{1.6cm} $\vv{b} =<2,-1,2>$\\
\begin{enumerate}
\item ${comp}_{\vv{b}}^{\vv{a}} \; = \; \dfrac{\vv{a} . \vv{b}}{|\vv{b}|} \; = \; \cfrac{4 - 1 - 2}{3} \; = \; \cfrac{1}{3}$
\item ${Proj}_{\vv{b}}^{\vv{a}} \; = \; \dfrac{\vv{a} . \vv{b}}{|\vv{b}|^2} \; \vv{b} \; = \; \cfrac{1}{9} \;<2,-1,2>$
\end{enumerate}
\end{example}
\noindent{\color{smalt(darkpowderblue)}\rule{\linewidth}{.2mm}}
\begin{example}
if ${Proj}_{\vv{b}}^{\vv{a}} \; = \;<2 , 1 , -2>$ , $\underbrace{\theta}_{\text{angle \; b/w} \; \vv{a} \; \& \;  \vv{b}}$ = $\cfrac{8}{15} \; \pi$\\
Find:
\begin{enumerate}
\item ${Proj}_{\vv{b}}^{2\vv{a}} \; = \; 2 \dfrac{\vv{a} . \vv{b}}{|\vv{b}|^2} \; \vv{b} \; = \; 2<2,1,-2> \; = \;<4 , 2 , -4>$
\item ${Proj}_{\vv{2b}}^{\vv{a}} \; = \; \dfrac{\vv{a} . 2\vv{b}}{4 |\vv{b}}| \; 2\vv{b} \; = \;<2,1,-2>$\\
\item ${Proj}_{-2\vv{b}}^{\vv{a}} \; = \; \dfrac{\vv{a} . (-2\vv{b})}{4 \; |\vv{b}|} \; (-2\vv{b}) \; = \; \dfrac{\vv{a} . \vv{b}}{|\vv{b}|^2} \; \vv{b} \; = \;<2 , 1 , 
-2>$\\
\item ${comp}_{\vv{b}}^{\vv{a}} \; = \; -3$
\end{enumerate}
\end{example}
\noindent{\color{smalt(darkpowderblue)}\rule{\linewidth}{.2mm}}
\begin{example}
Find $x$ such that the angle between $<2,1,-1>$, $<1,x,0>$ is $45^\circ$ \\
$\cos{45^\circ}=\cfrac{2+x-0}{\sqrt{6}\sqrt{1+x^2}}$~~~~~~
$|\cos{\theta}=\cfrac{\overrightarrow{a}.\overrightarrow{b}}{|\overrightarrow{a}||\overrightarrow{b}|}$\\
$\Leftrightarrow\cfrac{\sqrt{6}}{\sqrt{2}}\sqrt{1+x^2}=2+x$\\
$\Rightarrow3(1+x^2)=(2+x)^2$\\
$3+3x^2-4x-1=0$\\
$x=\cfrac{-b\pm\sqrt{b^2-4ac}}{2a}$
\end{example}
\noindent{\color{smalt(darkpowderblue)}\rule{\linewidth}{.2mm}}
\begin{example}
Find the two unit vectors that make an angle $60^\circ$ with $\overrightarrow{v}=<3,4>.$\\
\noindent\begin{minipage}{0.5\textwidth}
$\overrightarrow{u}=<a,b>$\\
$a^2+b^2=1\cdots(1)$\\
\end{minipage}
\noindent\begin{minipage}{0.5\textwidth}
$\cos{60}=\cfrac{\overrightarrow{u}.\overrightarrow{v}}{|\overrightarrow{u}|.|\overrightarrow{v}|}$\\
$\frac{1}{2}=\cfrac{3a+4b}{1.5}$\\
$3a+4b=\cfrac{5}{2}\cdots(2)$\\
\end{minipage}\\
\noindent\begin{minipage}{0.7\textwidth}
\includegraphics[width=7cm]{1w.jpg}\\
\end{minipage}
\noindent\begin{minipage}{0.5\textwidth}
$\theta^\circ=\tan\cfrac{4}{3}=x_\circ$\\
\end{minipage}
\end{example}
\noindent{\color{smalt(darkpowderblue)}\rule{\linewidth}{.2mm}}
53: the distance between the fine $ax+by+c=0$ and the point $(x_1,y_1)$ is\\
\noindent\begin{minipage}{0.5\textwidth}
$D=\cfrac{|ax_1+by_1+c|}{\sqrt{a^2+b^2}}$
\end{minipage}
\noindent\begin{minipage}{0.5\textwidth}
\includegraphics[width=5cm]{2w.jpg}
\end{minipage}
{\color{smalt(darkpowderblue)}Question:} Find the distance between $(-1,3)\&$ the line $3x-4y+5=0$
$$D=\cfrac{3(-2)-4(3)+5}{\sqrt{3^2+4^2}}=\cfrac{13}{5}$$
\noindent{\color{smalt(darkpowderblue)}\rule{\linewidth}{.2mm}}
\begin{example}
if $\overrightarrow{c}=|\overrightarrow{a}|\overrightarrow{b}+|\overrightarrow{b}|\overrightarrow{a},~\overrightarrow{a},\overrightarrow{b},\overrightarrow{c}$ not zero 
vectors\\
Show that $\overrightarrow{0}$ bisects $\overrightarrow{a}\&\overrightarrow{b}$\\
{\color{smalt(darkpowderblue)}\underline{Solution:}} \\
\noindent\begin{minipage}{0.7\textwidth}
$\cos{\alpha}=\cfrac{\overrightarrow{a}.\overrightarrow{c}}{|\overrightarrow{a}||\overrightarrow{c}|}=\cfrac{\overrightarrow{a}.[|\overrightarrow{a}|\overrightarrow{b}+|
\overrightarrow{b}\overrightarrow{a}}{|\overrightarrow{a}||\overrightarrow{c}|}$\\
$=\cfrac{\overrightarrow{a}.\overrightarrow{b}+|\overrightarrow{a}||\overrightarrow{b}|}{|\overrightarrow{c}|}$\\
\end{minipage}
\noindent\begin{minipage}{0.3\textwidth}
\includegraphics[width=5cm]{3w.jpg}
\end{minipage}\\
$\cos\beta=\cfrac{\overrightarrow{b}.\overrightarrow{c}}{|\overrightarrow{b}||\overrightarrow{c}|}=\cfrac{\overrightarrow{b.}[|\overrightarrow{a}|\overrightarrow{b}+|
\overrightarrow{b}|\overrightarrow{a}}{|\overrightarrow{b}||\overrightarrow{c}|}$\\
$=\cfrac{\overrightarrow{a}.\overrightarrow{b}+|\overrightarrow{a}||\overrightarrow{b}|}{|\overrightarrow{c}|}$\\
$\Rightarrow\alpha=\beta$
\end{example}
\begin{exercise}
Show that $proj_{\overrightarrow{b}}^{\overrightarrow{a}}.proj_{\overrightarrow{a}}^{\overrightarrow{b}}=(\overrightarrow{a}.\overrightarrow{b})\cos^2\theta$
\end{exercise}
\noindent{\color{smalt(darkpowderblue)}\rule{\linewidth}{.2mm}}
\begin{problem}
1,3,7,9,10,11,15,19,20,21,23,25,26,27,31,35,39,43,45,49,54,59
\end{problem}
\section{The Cross Product}
\begin{definition}
if $\overrightarrow{a}=<a_1,a_2,a_3>,~\overrightarrow{b}=<b_1,b_2,b_3>$ then the cross product of $\overrightarrow{a}\&\overrightarrow{b}$ is:
$$\overrightarrow{a}\times\overrightarrow{b}=\begin{vmatrix}
i&j&k\\
a_1&a_2&a_3\\
b_1&b_2&b_3
\end{vmatrix}$$
$$=\begin{vmatrix}
a_2&a_3\\
b_2&b_3
\end{vmatrix}i-
\begin{vmatrix}
a_1&a_3\\
b_1&b_3
\end{vmatrix}j+
\begin{vmatrix}
a_1&a_2\\
b_1&b_2
\end{vmatrix}k$$
$$=<a_2b_3-a_3b_2,-(a_1b_3-a_3b_1),a_1b_2-a_2b_1>$$
\end{definition} 
\begin{example}
If $\overrightarrow{a}=<1,2,-1>,~\overrightarrow{b}=<2,2,-3>$ Find $\overrightarrow{a}\times\overrightarrow{b}$\\
{\color{smalt(darkpowderblue)}\underline{Solution:}}\\
$\overrightarrow{a}\times\overrightarrow{b}=\begin{vmatrix}
i&j&k\\
1&2&-1\\
2&2&-3
\end{vmatrix}=(-4)i+1(j)+(-2)k=<-4,1,-2>.$
\end{example}
\begin{theorem}
$\overrightarrow{a}\times\overrightarrow{b}\bot\overrightarrow{a}~\&~\overrightarrow{a}\times\overrightarrow{b}\bot\overrightarrow{b}$\\
\end{theorem}
{\color{smalt(darkpowderblue)}\textbf{proof:}}
$(\overrightarrow{a}\times\overrightarrow{b}).\overrightarrow{a}=<a_2b_3-a_3b_2,-a_1b_3+a_3b_1,a_1b_2-a_2b_2>.<a_1,a_2,a_3>$\\
$=a_1a_2b_3-a_1a_3b_2-a_2a_3b_1+a_1a_3b_2-a_2a_3b_2=Zero$\\
$\Rightarrow(\overrightarrow{a}\times\overrightarrow{b}\bot\overrightarrow{a}).$
\begin{theorem}
$|\overrightarrow{a}\times\overrightarrow{b}|=|\overrightarrow{a}||\overrightarrow{b}|\sin{\theta}.~~~~0\leq\theta\leq180^\circ$\\
\end{theorem}
{\color{smalt(darkpowderblue)}\textbf{proof:}}
$|\overrightarrow{a}\times\overrightarrow{b}|^2=(a_2b_3-a_3b_2)^2+(a_1b_3-a_3b_1)^2+(a_1b_2-a_2b_1)^2$\\
$=a_2^2b_3^2+a_3^2b_2^2-2a_2a_3b_2b_3+a_1^2b_3^2+a_3^2b_1^2-2a_1a_3b_1b_3+a_1^2b_2^2+a_2^2b_1^2-2a_1a_2b_1b_2$\\
$=(a_1^2+a_2^2+a_3^2)(b_1^2+b_2^2+b_3^2)-(a_1b_1+a_2b_2+a_3b_3)^2$\\
$=|\overrightarrow{a}|^2|\overrightarrow{b}|^2-(\overrightarrow{a}\overrightarrow{b})^2$\\
$=|\overrightarrow{a}|^2|\overrightarrow{b}|^2-|\overrightarrow{a}|^2|\overrightarrow{b}|^2\cos\theta^2$\\
\noindent{\color{smalt(darkpowderblue)}\rule{\linewidth}{.2mm}}
{\color{smalt(darkpowderblue)}\textbf{Corollary:}}
if $\overrightarrow{a},\overrightarrow{b}$ non-zero vectors, then\\
$\overrightarrow{a}\parallel\overrightarrow{b}\iff \overrightarrow{a}\times\overrightarrow{b}=\overrightarrow{0}\iff 
|\overrightarrow{a}\times\overrightarrow{b}|=0\iff\overrightarrow{a}=c\overrightarrow{b}$ for some $c$\\
\begin{center}
    \includegraphics[width=4cm]{4w.jpg}
\end{center}
\begin{itemize}
\item $ {\overrightarrow{a}}\times{\overrightarrow{b}} =
\begin{vmatrix}
{i} & {j} & {k} \\
a_{1} & a_{2} & a_{3} \\
b_{1} & b_{2} & b_{3} 
\end{vmatrix}$
\item $a \times {\overrightarrow{b}} \perp {\overrightarrow{a}} \& {\overrightarrow{a}} \times {\overrightarrow{b}} \perp {\overrightarrow{b}}$
\item $\ \mid{\overrightarrow{a}}\times{\overrightarrow{b}}\mid = \mid{\overrightarrow{a}}\mid  \mid{\overrightarrow{b}}\mid sin \theta$ \hspace{2cm} $g \in [0,\pi ]$
\item $ \mid{\overrightarrow{a}}\times{\overrightarrow{b}} =$ area of the parallelogram that determined by ${\overrightarrow{a}}\times{\overrightarrow{b}}$ 
\end{itemize}
\noindent{\color{smalt(darkpowderblue)}\rule{\linewidth}{.2mm}}
%===========================================================
\begin{example}
find the area of the triangle with vertices\\
{\color{smalt(darkpowderblue)}{\underline{Solution}}} :
\begin{itemize}
\item $P(2,1,3)$
\item $Q(1,-1,1)$
\item $R(3,2,-2)$
\item ${\overrightarrow{PQ}} = < -1,-2,-2 > $
\item  ${\overrightarrow{PR}} = < 1,1,-5 >$
\item $A = \mid{\overrightarrow{PQ}}\times{\overrightarrow{PR}}\mid$
\end{itemize}
${\overrightarrow{PQ}}\times{\overrightarrow{PR}}$=
$\begin{vmatrix}
{i} & {j} & {k} \\
{-1} & {-2} & {-2} \\
{1} &{1} & {-5} \\
\end{vmatrix}$
$= <12,-7,1>$\\ \\
$\Rightarrow A =   \mid{\overrightarrow{PQ}}\times{\overrightarrow{PR}}\mid \\ 
=  \sqrt{144+49+1} \\
= \sqrt{194}$ \\
\end{example}
\noindent{\color{smalt(darkpowderblue)}\rule{\linewidth}{.2mm}}
%===================================================================
\begin{example}
Find 
\begin{enumerate}
\item $i \times i$
\item $i \times j $
\item $k \times j$
\item $<-1,-2,-2> \times <1,1,-5> $
\end{enumerate}
{\color{smalt(darkpowderblue)}{\underline{Solution}}} :
\begin{enumerate}
\item $ {\overrightarrow{0}} = <0,0,0>$
\item $ k$
\item $-i$
\item $= (-i -2j -2k) \times (i + j -5k)$\\
$ = -k -5j +2k +10i -2j +2i $\\
$= <12,-7,1>$
\end{enumerate}
\end{example}
\noindent{\color{smalt(darkpowderblue)}\rule{\linewidth}{.2mm}}
%==========================================================
\begin{example}
True or False :
\begin{enumerate}
\item ${\overrightarrow{a}}\times{\overrightarrow{b}} = {\overrightarrow{b}}\times{\overrightarrow{a}}$
\item  $({\overrightarrow{a}}\times{\overrightarrow{b}}) \times {\overrightarrow{c}} = {\overrightarrow{a}}\times ({\overrightarrow{b}}\times \overrightarrow{c}) $
\item  $i\times(i \times j) = i \times k = -j$\\
$(i \times i) \times j = \overrightarrow{0} \times i = \overrightarrow{0}$
\end{enumerate}
{\color{smalt(darkpowderblue)}{\underline{Solution}}} :
\begin{enumerate}
\item False : ~ $ i \times j = k \neq j \times i = -k $
\item False \begin{itemize}
\item $ i\times(i \times j) = i \times k = -j$
\item $(i \times i) \times j = \overrightarrow{0} \times i = \overrightarrow{0}$
\end{itemize}
\end{enumerate}
\end{example}
\noindent{\color{smalt(darkpowderblue)}\rule{\linewidth}{.2mm}}
{\color{smalt(darkpowderblue)}{\underline{Properties}}} : 
\begin{enumerate}
\item  ${\overrightarrow{a}}\times{\overrightarrow{b}} = -{\overrightarrow{b}}\times{\overrightarrow{a}}$
\item $(c{\overrightarrow{a}})\times{\overrightarrow{b}} = {\overrightarrow{a}}\times (c{\overrightarrow{b}}) = c({\overrightarrow{a}}\times{\overrightarrow{b}}) $
\item ${\overrightarrow{a}}\times({\overrightarrow{b}} + \overrightarrow{c}) = {\overrightarrow{a}}\times{\overrightarrow{b}} + 
{\overrightarrow{a}}\times{\overrightarrow{c}}$
\item $(\overrightarrow{a} + \overrightarrow{b}) \times \overrightarrow{c} = {\overrightarrow{a}}\times{\overrightarrow{c}} + {\overrightarrow{b}}\times{\overrightarrow{c}}$
\item $\overrightarrow{a} \cdot (\overrightarrow{b} \times \overrightarrow{c}) = (\overrightarrow{a} \times \overrightarrow{b}) \cdot \overrightarrow{c}$
\item $\overrightarrow{a} \times (\overrightarrow{b} \times \overrightarrow{c}) = (\overrightarrow{a} \cdot \overrightarrow{c}) \overrightarrow{b} - (\overrightarrow{a} 
\cdot \overrightarrow{b})\overrightarrow{c}$
\end{enumerate}
{\color{smalt(darkpowderblue)}{Pf(5)}} if :
\begin{itemize}
\item $\overrightarrow{a} = <a_{1} , a_{2} , a_{3} >$
\item $\overrightarrow{b} = < b_{1} , b_{2} , b_{3} >$
\item $\overrightarrow{c} = <c_{1} , c_{2} , c_{3}>$
\end{itemize}
{\color{smalt(darkpowderblue)}{\underline{L.H.S}}}\\ = $\overrightarrow{a} \cdot (\overrightarrow{b} \times \overrightarrow{c})$\\
$= <a_{1},a_{2},a_{3}> \cdot <b_{2}c_{3} - b_{3}c_{2} , b_{3}c_{1} - b_{1}c_{3} , b_{1}c_{2} -b_{2}c_{1} >$\\
$= a_{1}b_{2}c_{3} - a_{1}b_{3}c_{2} + a_{2}b_{3}c_{1} - a_{2}b_{1}c_{3} + a_{3}b_{1}c_{2} - a_{3}b_{2}c_{1}$\\
{\color{smalt(darkpowderblue)}{\underline{R.H.S}}}\\ = $(\overrightarrow{a} \times \overrightarrow{b}) \cdot \overrightarrow{c}$\\
$= <a_{2}b_{3} - a_{3}b_{2} , a_{3}b_{1} - a_{1}b_{3} , a_{1}b_{2} - a_{2}b_{2} > \cdot <c_{1} , c_{2}, c_{3} >\\
= a_{2}b_{3}c_{1} - a_{3}b_{2}c_{1} + a_{3}b_{1}c_{2} - a_{1}b_{3}c{2} + a_{1}b_{2}c_{3} + a_{2}b_{1}c_{3}\\
\Rightarrow$ {\color{smalt(darkpowderblue)}{L.H.S }}={\color{smalt(darkpowderblue)}{ R.H.S}} \\
\noindent{\color{smalt(darkpowderblue)}\rule{\linewidth}{.2mm}}
\begin{example}
If  $\overrightarrow{a}\cdot(\overrightarrow{b} \times \overrightarrow{c}) = 2$ ,\\ Find $2 \overrightarrow {b} \cdot (\overrightarrow{a} \times 2 \overrightarrow{c})$\\
{\color{smalt(darkpowderblue)}{\underline{Solution}}} : \\
$ = 4 \overrightarrow{b} \cdot (\overrightarrow{a} \times \overrightarrow{c})\\
= 4 (\overrightarrow{a} \times \overrightarrow{c}) \cdot \overrightarrow{b}\\
= 4 \overrightarrow{a} \cdot (\overrightarrow{c} \times \overrightarrow{b})\\
= -4 \overrightarrow{a} \cdot (\overrightarrow{b} \times \overrightarrow{c})\\ =-8$\end{example}
\noindent{\color{smalt(darkpowderblue)}\rule{\linewidth}{.2mm}}
$\overrightarrow{a} \cdot (\overrightarrow{b} \times \overrightarrow{c}) = (\overrightarrow{a} \times \overrightarrow{b}) \cdot \overrightarrow{c} =$ $\begin{vmatrix}
a_{1} & a_{2} & a_{3} \\
  b_{1} & b_{2} & b_{3} \\
  c_{1} & c_{2} & c_{3} 
 \end{vmatrix}$\\
 triple product $\overrightarrow{a} \cdot (\overrightarrow{b} \times \overrightarrow{c})$\\
{\color{smalt(darkpowderblue)}{ volume of the parallelepiped }}:
$$\overrightarrow{\nu}= \mid \overrightarrow{a} \cdot (\overrightarrow{b} \times \overrightarrow{c}) \mid$$
\noindent{\color{smalt(darkpowderblue)}\rule{\linewidth}{.2mm}}
 \begin{example}
  find the volume of the parallelepiped that determine by \\
 $\overrightarrow{a} = <1,2,-1>$\\
 $\overrightarrow{b} = <2,1,1>$\\
 $\overrightarrow{c} = <3,2,-2>$\\
 {\color{smalt(darkpowderblue)}{\underline{Solution}}} :
 \begin{align*}
      \overrightarrow{\nu}=\mid \overrightarrow{a} \cdot (\overrightarrow{b} \times \overrightarrow{c}) \mid 
      & = 
\begin{Vmatrix}
 1 & 2 & -1 \\
 2 & 1 & 1 \\
 3 & 2 & -2 
 \end{Vmatrix} \\
 & =\mid 1(-4) -2(-7) + -1(1) \mid \\
 &= 9 
 \end{align*}
if $\overrightarrow{a}\cdot(\overrightarrow{b}\times \overrightarrow{c}) = 0$ then we say that $\overrightarrow{a} , \overrightarrow{b}$ and $\overrightarrow{c}$ are called 
\textbf{\underline{coplaner}}
\end{example}
\noindent{\color{smalt(darkpowderblue)}\rule{\linewidth}{.2mm}}
%================================================================
\begin{example}
 show that the following vectors are coplaner \\
$\overrightarrow{a} = < 2,1,-1 >$\\
$\overrightarrow{b} = < -1,3,2 >$\\
$\overrightarrow{c} = < 0,7,3 >$\\
 {\color{smalt(darkpowderblue)}{\underline{Solution}}} : 
 \begin{align*}
  \overrightarrow{a} \cdot \overrightarrow{b} \times \overrightarrow{c} = &
 \begin{vmatrix}
 2 & 1 & -1 \\
 -1 & 3 & 2 \\
 0 & 7 & 3 
 \end{vmatrix} \\
 & =2(-5) -1(-3) + -1(-7) \\
 & = -10 +3 +7 = 0 
 \end{align*}
Thus \textbf{coplaner}
\end{example}
%====================================================================
\noindent{\color{smalt(darkpowderblue)}\rule{\linewidth}{.2mm}}
\begin{problem}
1,5,7,9,13,18,19,27,29,31,34,35,38,43
\end{problem}
\section{Equation of linear $\&$ planes}
{\color{smalt(darkpowderblue)}{\underline{ in 3D :}}}\\
 To determine a line , we need :
 \begin{enumerate}
     \item  point $(x_\cdot , y_\cdot , z_\cdot )$
 \item parallel vector $\overrightarrow{\nu} = <a,b,c>$
 \end{enumerate}
 find it is equations ! \\
 note that $\overrightarrow{\nu} \parallel \overrightarrow{r} $\\
 $\overrightarrow{\nu} = t\overrightarrow{a} ~, t \in R $\\ {\color{smalt(darkpowderblue)}{vector equation of the line .}}\\
 $< x-x_0 , y-y_0 , z-z_0 > = < ta , tb ,tc >$\\
 parametric equations of the line :
 \begin{itemize}
     \item $x = x_\cdot + at $
\item $y = y_\cdot + bt $
\item $z = z_\cdot + ct $
 \hspace{1cm} $ -\infty < t < \infty $
  \end{itemize}
\noindent{\color{smalt(darkpowderblue)}\rule{\linewidth}{.2mm}}
%======================================================================
 \begin{example}
 \begin{enumerate}
     \item  find the parametric equations of the line that passes through the point $(2,1,3) \&$ parallel vector $\overrightarrow{\nu} = <2,1,-1>$ .
      \item Find the point on the line
      \item does the point (0,0,5) lie on the line ?
 \end{enumerate}
  {\color{smalt(darkpowderblue)}{\underline{Solution}}} : 
\begin{enumerate}
    \item $ x = 2 + 2t $\\
 $ y = 1 + t$\\
 $ z = 3 - 2t $ \hspace{1cm}
 $ -\infty < t < \infty$
 \item $ t = 2 \Rightarrow (6,3,-1) $\\
 $ t = \cfrac{7}{2} \Rightarrow (9,\cfrac{9}{2} , -4)$
 \item $ 0 = 2+2t \Rightarrow t =-1 $\\
 $ 0=1+t \Rightarrow t=-1 $\\
 $ 5=3-2t \Rightarrow t=-1 $ \hspace{1cm} {\color{smalt(darkpowderblue)}{yes}}.
\end{enumerate}
\end{example} 
%==============================================================
\noindent{\color{smalt(darkpowderblue)}\rule{\linewidth}{.2mm}}
\begin{itemize}
    \item  $ x = x_0 +at \Rightarrow t = \cfrac{x - x_0}{a} \hspace{1cm }a \neq 0 $
    \item  $ y = y_0 +bt \Rightarrow t = \cfrac{y - y_0}{b} \hspace{1cm }b \neq 0 $
    \item  $ z = z_0 +ct \Rightarrow t = \cfrac{z - z_0}{c} \hspace{1cm }c \neq 0 $
\end{itemize}
 So, if
 \begin{itemize}
     \item $ a\neq 0$
     \item $ b\neq 0$
     \item $c\neq 0$ 
 \end{itemize}
 $$ \Rightarrow
  \cfrac{x-x_0}{a} = \cfrac{y-y_0}{b} = \cfrac{z-z_0}{c}
 \hspace{1cm} a\neq 0 ~~ , b\neq 0 ~~ , c\neq 0 $$
 {\color{smalt(darkpowderblue)}{symmetric equation}}
 $$\cfrac{x-x_0}{a} = \cfrac{y-y_0}{b} = \cfrac{z-z_0}{c}$$
 if $a=0$
 $$ x=x_0 ~~, \cfrac{y-y_0}{b} = \cfrac{z-z_0}{c}$$
 \noindent{\color{smalt(darkpowderblue)}\rule{\linewidth}{.2mm}}
%========================================================================
\begin{example}
\begin{enumerate}
    \item find the equation of the line that passes thorough the point $P(1,2,-1) \& Q(3,1,2)$
    \item  find where the line intersected the xy-plane !
\end{enumerate}
  {\color{smalt(darkpowderblue)}{\underline{Solution}}} : 
  \begin{enumerate}
      \item We need 
  \begin{enumerate}
      \item point P(1,2,-1)
      \item $\overrightarrow{\nu} = <2,-1,3>$
      \end{enumerate}

\textbf{Parametric eq.s}
$$x = 1+2t$$               
$$y= 2-t$$
$$z = -1+3t$$
\textbf{Symmetric eq.}
$$\cfrac{x-1}{2} = \cfrac{y-2}{-1} = \cfrac{z+1}{3}$$
\item  z= 0  
\begin{itemize}
    \item $\Rightarrow \cfrac{x-1}{2} = \cfrac{1}{3} \\ \Rightarrow x-1 =\cfrac{2}{3} \\ \Rightarrow x = \cfrac{5}{3}$ 
    \item $\cfrac{y-2}{-1} = \cfrac{1}{3} \rightarrow y-2 = \cfrac{-1}{3} \rightarrow y=\cfrac{5}{3}$
    \item $(\cfrac{5}{3} , \cfrac{5}{3} , 0)$
\end{itemize}
 \end{enumerate}
 \end{example}
 \noindent{\color{smalt(darkpowderblue)}\rule{\linewidth}{.2mm}}
%=======================================================================
\begin{example}
Find the parametric equations of the line that passes through the point $(-2,1,1)$ \& parallel to the line :\\$L_1 = \cfrac{x-2}{1} = \cfrac{2-y}{1} = \cfrac{2z+1}{1}.$\\
  {\color{smalt(darkpowderblue)}{\underline{Solution}}}: We need 
\begin{enumerate}
    \item  point $(-2,1,1)$
    \item $\overrightarrow{\nu} = <1,-1,\cfrac{1}{2}>$
\end{enumerate}
\begin{align*}
    &x=2+t \\
    &y=1-t \\
    &z=1+\cfrac{1}{2}t \hspace{1cm} t \in R
\end{align*}
\end{example}
\noindent{\color{smalt(darkpowderblue)}\rule{\linewidth}{.2mm}}
{\color{smalt(darkpowderblue)}{\color{smalt(darkpowderblue)}\underline{line}}} 
\begin{enumerate}
    \item point $(x_\cdot , y_\cdot , z_\cdot )$
    \item  parallel vector $\overrightarrow{\nu} = <a,b,c>$
\end{enumerate}
{\color{smalt(darkpowderblue)}{\color{smalt(darkpowderblue)}\underline{Parametric equation}}} 
\begin{itemize}
    \item $x=x_0 + at$
    \item $y=y_0 + bt$ 
    \item $z=z_0 +ct$ \hspace{1cm}$ -\infty < t < \infty $
\end{itemize}
{\color{smalt(darkpowderblue)}{\color{smalt(darkpowderblue)}\underline{Symmetric equations}}} 
\begin{itemize}
    \item $ \cfrac{x-x_0}{a} = \cfrac{y-y_0}{b} = \cfrac{z-z_0}{c}$
\end{itemize} 
\begin{remark}
  two lines are parallel iff their vector are parallel 
\end{remark}
\begin{definition}
 two lines are called skew if they are not parallel $\&$ they do not intersect .
\end{definition}
 \noindent{\color{smalt(darkpowderblue)}\rule{\linewidth}{.2mm}}
%=====================================================================
\begin{example}
 show that the following lines are skew \\
$L_1 : x = 1+t , y=-2+3t , z=4-t / \overrightarrow{\nu_{1}} = <1,3,-1> $\\
$L_2 : x = 2s , y=3+s , z=-3+4s / \overrightarrow{\nu_{2}} = <2,1,4>$\\
{\color{smalt(darkpowderblue)}\underline{Solution}}\\
$\nu_{1} \neq \nu_{2} \Rightarrow L_{1} \neq L_{2} \hspace{1cm}(L_{1} \& L_2 $ are not parallel )\\
$1+t = 2s $ \hspace{4cm} $t-2s=-1 \rightarrow (1) $\\
$-2+2t = 3+s$ \hspace{3.1cm}$3t-s=5 \rightarrow (2) $\\
$4-t = -3 +4s $\hspace{3.1cm} $-t -4s=-7 \rightarrow (3)$\\
{Solve ($1) \& (3$)} \\
$ 0-6s=-8 \Rightarrow s=\cfrac{8}{6} = \cfrac{4}{3}$\\
$t=-1+2s \Rightarrow t=-1+\cfrac{8}{3} = \cfrac{5}{3}$\\
$s=\cfrac{4}{3} ~~, t=\cfrac{5}{3}$\\
in Equation 2 \\ $ 3\cfrac{5}{3} - \cfrac{4}{3} \neq  5  \Rightarrow 5-\cfrac{4}{3} \neq 5$\\
$L_{1}~ \& ~ L_2$  do not intersect $\Rightarrow L_{1}\& L_2$ are skew \end{example}
 \noindent{\color{smalt(darkpowderblue)}\rule{\linewidth}{.2mm}}
%========================================================================
{\color{smalt(darkpowderblue)}\underline{Planes}} : to determine a plane we need \begin{enumerate}
    \item point $(x_0 , y_0 z_0 )$
    \item normal vector $\overrightarrow{n} = <a,b,c>$ \hspace{2cm} Note that $\overrightarrow{\nu} \perp \overrightarrow{n}$\\
$\Rightarrow \overrightarrow{\nu} \cdot \overrightarrow{n} = 0$
\end{enumerate}
$\Rightarrow <x-x_0 , y-y_0 , z-z_0> \cdot <a,b,c> = 0$\\
$\Rightarrow  a(x-x_0) + b(y-y_0) +c(z-z_0) = 0$\\
$\Rightarrow  ax + by + cz + -(ax_0 + by_0 +cz_0) = 0$\\
$\Rightarrow  ax + by +cz +d + 0 $\\
$\Rightarrow  d = -(ax_0 + by_0 + cz_0 )$\\
 \noindent{\color{smalt(darkpowderblue)}\rule{\linewidth}{.2mm}}
\begin{example}
 \begin{enumerate}
     \item find the equation of the plane that passes through the point $(1,-1,3) \&$  normal vector $\overrightarrow{\nu} = <2,1,-1> .$
     \item find two points on the plane .
 \end{enumerate}
{\color{smalt(darkpowderblue)}\underline{Solution}}
\begin{enumerate}
    \item point p(1,-1,3) 
    \item normal vector $\overrightarrow{\nu} = <2,1,-1> .$\\
$2(x-1) + 1(y+1) + -1(z-3) = 0 $\\
$ 2x +y -z +6 = 0$\\
2) (0,0,6)
(-3,0,0)
(0,-6,0) 
\end{enumerate}
\end{example} 
 \noindent{\color{smalt(darkpowderblue)}\rule{\linewidth}{.2mm}}
\begin{example}
 find the equation of the plane that passes through the points
$P(2,1,-2) \hspace{.4cm} Q(1,1,-1)\hspace{.4cm} R(3,-2,1)$ \\
{\color{smalt(darkpowderblue)}\underline{Solution}}
\begin{enumerate}
    \item  Point $(2,1,-2)$
    \item $\overrightarrow{n} = \overrightarrow{PQ} \times \overrightarrow{PR}$\\
$=<3,4,3>$\\
$=3x + 4y  + 3z + -4\\ = 0$\\
$\overrightarrow{PQ} <-1,0,1>$\\
$\overrightarrow{PR} = <1,-3,3>$
\begin{align*}
\overrightarrow{n} = \overrightarrow{PQ} \times \overrightarrow{PR}
&=\begin{vmatrix}
 i & j & k \\
  -1 & 0 & 1 \\
  1 & -3 & 3 
 \end{vmatrix}\\
 & =<3,4,3>.
 \end{align*}
\end{enumerate}
\end{example}
 \noindent{\color{smalt(darkpowderblue)}\rule{\linewidth}{.2mm}}
 {{\color{smalt(darkpowderblue)}{\underline{Plane}}}}
\begin{itemize}
    \item $ p(x_0 , y_0 , z_0)$
    \item $\overrightarrow{n} = <a,b,c>$\\
$ax+by+cz+d = 0$\\
$d=-(ax_\cdot + by_\cdot +cz_\cdot)$
\end{itemize}
\noindent{\color{smalt(darkpowderblue)}\rule{\linewidth}{.2mm}}
\begin{example}
Find the equation of the plane that passes through the point 
p(1,2,1), Q(2,3,2), R(-1,-1,3)\\
{{\color{smalt(darkpowderblue)}{\underline{Solution}}}} \\
$ax+by+cz+d=0 $ 
\begin{itemize}
    \item if $d\neq 0 $ \\ $Ax+By+Cz+1=0 \\ 
    A+2B+C+1=0 \\ 2A+3B+2C+1=0 \\ -A-B-3C+1=0 $ \hspace{1cm} (Rejected)
    \item $d=0 \\ ax+by+cz+1=0$
\end{itemize}
\end{example}
\noindent{\color{smalt(darkpowderblue)}\rule{\linewidth}{.2mm}}
\begin{example}
Find the equation of the plane that passes through the point (1,-1,2) \& contains the line \\
$L_{1} : \cfrac{x-1}{2} = \cfrac{y+1}{2} = \cfrac{z-1}{3}. \overrightarrow{\nu} = <2,3,1>$\\
{{\color{smalt(darkpowderblue)}{\underline{Solution}}}}
\begin{enumerate}
    \item  Point $(1,-1,2)$
    \item  normal vector 
    \begin{align*}
&\overrightarrow{n} = \overrightarrow{r} \times \overrightarrow{\nu}\\
&=\overrightarrow{RQ} \times \overrightarrow{\nu}\\
&=<0,0,1> \times <2,2,3>\\
&-2x+2y+0z+4=0\\
&\Rightarrow x-y-2=0
\end{align*}
\end{enumerate}
\end{example} 
\noindent{\color{smalt(darkpowderblue)}\rule{\linewidth}{.2mm}}
\begin{example}
find the equations of the line of intersection of the following planes .\\
$P_{1} : 2x-y+z_3 = 0 . \overrightarrow{n_{1}} = <2,-1,1>$\\
$P_{2} : x-3y-z-1=0 . \overrightarrow{n_{2}}=<1,-3,-1>$\\
{{\color{smalt(darkpowderblue)}{\underline{Solution}}}}
\begin{itemize}
    \item Point :$ P(0,-1,2) $
    \item Parallel vector \\ 
    $\overrightarrow{\nu} = \overrightarrow{n_{1}} \times \overrightarrow{n_{2}}$
=$\begin{vmatrix}
 {i} & {j} & {k} \\
  2 & -1 & 1 \\
  1 & -3 & -1 
 \end{vmatrix}$\\
 $=<4,3,-5>$ 
 let x=0 \\ 
 $-y +z=3$\\
 +\\
 $-3y-z=1$\\
 $\Rightarrow -4y = 4 \Rightarrow y =-1 \\ \Rightarrow z=2$\\
 $x=0+4t$\\
 $y=-1+3t$\\
 $z=2-5t$\\
\end{itemize} 
\noindent{\color{smalt(darkpowderblue)}\rule{\linewidth}{.2mm}}
{{\color{smalt(darkpowderblue)}{\underline{Solution 2}}}} : Pick two points on the line :
 \begin{itemize}
     \item  $P(0,-1,2) $
     \item $Q(\cfrac{4}{3},0,\cfrac{1}{3})$
 \end{itemize}
  Let $y=0$  
  $$2x+z=3 $$
$$+$$
$$x-z=1 $$
$$\Rightarrow 3x=4$$  $$\Rightarrow x=\cfrac{4}{3}$$
$$\Rightarrow z=\cfrac{4}{3} -1 = \cfrac{1}{3}$$
$\overrightarrow{\nu} = \overrightarrow{PQ} = <\cfrac{4}{3},1,\cfrac{-5}{3}>$
\begin{itemize}
    \item $x=0+\cfrac{4}{3} t$
    \item $y=-1+t$
    \item $z=2-\cfrac{5}{3} t$
\end{itemize}
\noindent{\color{smalt(darkpowderblue)}\rule{\linewidth}{.2mm}}
{{\color{smalt(darkpowderblue)}{\underline{Solution 3 }}}}: \\
$2x-y+z-3 =0$\\   
$x-3y-z-1=0$\\   let $x=t $\\ 
$\Rightarrow 2t-y+z-3 = 0$ \\
$\Rightarrow t-3y-z-1=0$\\
$\downarrow$\\
$-y+z=3-2t$\\
+\\
$-3y-z=1-t$\\ \\
$\Rightarrow -4y = 4-3t\\ \Rightarrow y=-1+\cfrac{3}{4} t$\\
$-3(-1+\cfrac{3}{4}t) -z = 1-t$\\
$z=3-\cfrac{9}{4}t - 1+t$\\
$z=2-\cfrac{5}{4} t$
\end{example} 
\begin{remark}
  \begin{itemize}
      \item  two plane are parallel if their normal vectors are parallel .
      \item  the angle blw too plane is defined to be the  active angle blw $\overrightarrow{n_{1}} \& \overrightarrow{n_{2}}.$
  \end{itemize}
\end{remark}
%===================================================================
\begin{example}
find the angle btw the following plane : 
$ 2x-2y=z-1 = 0 \\
x=3y-z=7 = 0 ,.$
{{\color{smalt(darkpowderblue)}{\underline{Solution}}}}: 
\begin{itemize}
    \item $ \overrightarrow{n_{1}} = <2,-2,1>$
    \item $\overrightarrow{n_{2}} = <1,3,-1>$
\end{itemize}
$\theta : cos\theta = \cfrac{\overrightarrow{n_1} \cdot \overrightarrow{n_2}}{\mid \overrightarrow{n_1}\mid \mid \overrightarrow{n_2}\mid} = \cfrac{2-6-1}{3\sqrt{11}} = -\cfrac{5}{3\sqrt{11}} \approx 120.1\\
\Rightarrow \alpha = 59.9$\end{example}
\noindent{\color{smalt(darkpowderblue)}\rule{\linewidth}{.2mm}}
\begin{example}
 find the intersection btw the following line :\\
$P:2x-2y+z_1 = 0 $\\
$L: x=1+t , y=1-t , z=t .$\\
{{\color{smalt(darkpowderblue)}{\underline{Solution}}}}:\\ 
 $2(1+t) -2(1-t) +t-1 = 0$\\
$2+2t-2+2t+t=1$\\
$5t = 1\Rightarrow t=\cfrac{1}{5}$\\
$x=1+\cfrac{1}{5} \Rightarrow \cfrac{6}{5}$\\
$y=1-\cfrac{1}{5} \Rightarrow \cfrac{4}{5}$\\
$z=\cfrac{1}{5}$
\end{example}
\noindent{\color{smalt(darkpowderblue)}\rule{\linewidth}{.2mm}}
{\color{smalt(darkpowderblue)}{\underline{Distances :}}}\\
$$\cfrac{1}{2}*|\overrightarrow{QR}|*D=\cfrac{1}{2}*|\overrightarrow{QR}*\overrightarrow{PQ}|$$
$$D=\cfrac{|\overrightarrow{QR}|*|\overrightarrow{PQ}|}{|QR|}$$
OR \\
$$D=\cfrac{|\overrightarrow{v}*\overrightarrow{PQ}|}{\overrightarrow{|v|}}$$
\begin{example}
Find the distance below $(1,2-1) \&$ the line \\ 
$x=1+t \\
y=1-t \\
z=t$\\
{\color{smalt(darkpowderblue)}{\underline{Solution :}}}
Pick two points on the line \\
$t=0 \Rightarrow ~~~~~ Q(1,1,0) \\
t=1 \Rightarrow ~~~~  R(2,0,1)$\\
$D=\cfrac{|\overrightarrow{QR}|*|\overrightarrow{PQ}|}{|QR|}$\\ 
$\overrightarrow{PQ}=<0,-1,1> \\
\overrightarrow{QR} =<1,-1,1>$ \\
$\begin{array}{rcl}
\overrightarrow{PQ}*\overrightarrow{QR} & =
\begin{vmatrix}
 i& j&k  \\
0 &  -1&1  \\
 1&  -1&1 
\end{vmatrix}
\end{array}$\\
$=\cfrac{\sqrt{0+1+1}}{\sqrt{1+1+1}}=\sqrt{\cfrac{2}{3}}$
\end{example}
\noindent{\color{smalt(darkpowderblue)}\rule{\linewidth}{.2mm}}
\begin{example}
Find the distance below the point $(x,y,z) \&$ the plane \\ $ax+by+cz=0 $\\
{\color{smalt(darkpowderblue)}{\underline{Solution :}}}\\
$\begin{array}{rcl}
D & = & \left| \mathrm{Comp}_{\overrightarrow{n}}^{\overrightarrow{r}} \right| \\
 & = & \cfrac{|\overrightarrow{r}.\overrightarrow{n}|}{|\overrightarrow{n}|} \hspace{5cm} \overrightarrow{r}=<x_0-x_1,y_0-y_1,z_0-z_1> \\
 & = & \cfrac{|a(x_1-x_0)+b(y_1-y_0)+c(z_1-z_0)|}{\sqrt{a^2+b^2+c^2}} \hspace{1.7cm} \overrightarrow{n}=<a,b,c> \\
 & = & \cfrac{|ax_1+by_1+cz_1+-(ax_0+by_0+cz_0)|}{\sqrt{a^2+b^2+c^2}} \\
D& = & \cfrac{|ax_1+by_1+cz_1+d|}{\sqrt{a^2+b^2+c^2}}
\end{array}$
\end{example} 
\noindent{\color{smalt(darkpowderblue)}\rule{\linewidth}{.2mm}}
\begin{example}
Find the distance between the point $(1,2,-1)$ and the plane $2x+2y-z-3=0$\\
{\color{smalt(darkpowderblue)}{\underline{Solution :}}} 
$\begin{array}{rcl}
 D& = & \cfrac{|ax_1+by_1+cz_1+d|}{\sqrt{a^2+b^2+c^2}} \\
 & = & \cfrac{|2(1)+2(2)-1(-1)-3|}{\sqrt{9}}\\
 & = & \cfrac{4}{3}
\end{array}$
\end{example}

\noindent{\color{smalt(darkpowderblue)}\rule{\linewidth}{.2mm}}
\begin{example}
Find the distance between the following planes \\
$P_1 : x+y-z+1 =0\\
P_2 : 2x-2y+3z+7=0 $\\
{\color{smalt(darkpowderblue)}{\underline{Solution :}}}
$\overrightarrow{n_1}=<1,1,-1> \\
\overrightarrow{n_2}=<2,-2,3> \\
\Rightarrow \overrightarrow{n_1}
\nparallel \overrightarrow{n_2} \\
\Rightarrow P_1 \nparallel P_2 \\
\Rightarrow P_1 ~ and ~ P_2  are intersected \\
\Rightarrow D=0$
\end{example}

\noindent{\color{smalt(darkpowderblue)}\rule{\linewidth}{.2mm}}
\begin{example}
Find the distance between the following plane\\ 
$P_1 : x-2y+2z+3=0 \\ 
P_2 : -2x+4y-4z-5=0 $\\
{\color{smalt(darkpowderblue)}{\underline{Solution :}}} \\ 
$\overrightarrow{n_1}=<1,-2,2> \\
\overrightarrow{n_2}=<-2,4,-4>$ \\
Thus $ \overrightarrow{n_1} \parallel 
\overrightarrow{n_2} $\\
Pick any point on $P_1 \Rightarrow p(-3,0,0) \\
D=\cfrac{|-2(-3)+4(0)-4(0)-5|}{\sqrt{4+16+16}} =\cfrac{1}{6}$
\end{example}
\noindent{\color{smalt(darkpowderblue)}\rule{\linewidth}{.2mm}}

\begin{example}
Find the distance between the line and plane \\
$L : x=1+t, ~~~~ y=1-t, ~~~~ z=t \\
P : 2x-y+z+3=0$\\
{\color{smalt(darkpowderblue)}{\underline{Solution (1):}}} \\ 
$\overrightarrow{v}=<1,-1,1> \\ \overrightarrow{n}=<2,-1,1> \\ 
P \parallel L \Longleftrightarrow \overrightarrow{v} \bot \overrightarrow{n} \Longleftrightarrow \overrightarrow{n} . \overrightarrow{v}=0$\\ 
Note that \\
$\begin{array}{rcl}
\overrightarrow{n} . \overrightarrow{v} & = & 2+1+1 \\
 & = & 4  \\
 & \neq & 0
\end{array}$ \\
$\Rightarrow \overrightarrow{n} \not\perp \overrightarrow{v} \\
\Rightarrow P \nparallel L \\
\Rightarrow D=0$ \\ \\
{\color{smalt(darkpowderblue)}{\underline{Solution (2):}}} \\ 
Try to find an intersection point \\ 
$2(1+t)-(1-t)+t+3=0 \\
2+2t-1+t+t+3=0 \\ 4t=-4 \Rightarrow t=-1$ \\
P \& L are intersecting \\
$\Rightarrow D=0$
\end{example}

\noindent{\color{smalt(darkpowderblue)}\rule{\linewidth}{.2mm}}
\begin{example}
Find the distance between the following plane $\&$ line \\$ L : x=1-t, ~~~~ y=t, ~~~~ z=2-t \\
P : 2x+y-z+3=0 $\\
{\color{smalt(darkpowderblue)}{\underline{Solution :}}} \\
$\overrightarrow{v}=<-1,1,-1> \\
\overrightarrow{n}=<2,1,-1> \\
\overrightarrow{n} . \overrightarrow{v}=0 \Rightarrow P \parallel L$\\ pick a point on the line ~~ $t=0 \Rightarrow (1,0,2) \\ $
$\begin{array}{rcl}
D & = & \cfrac{|2+0-2+3|}{\sqrt{4+1+4}}  \\
 & = & \cfrac{3}{\sqrt{6}} \\
 & = & \sqrt{\cfrac{3}{2}}
\end{array}$
\end{example}
\noindent{\color{smalt(darkpowderblue)}\rule{\linewidth}{.2mm}}

\begin{example}
Find the distance between the following lines \\
$L_1 : x=1-t ,~~~~ y=1+t, ~~~~ z=t \\
L_2 : x=-1+2t ,~~~~ y=-2t ,~~~~ z=1-2t $  \\
{\color{smalt(darkpowderblue)}{\underline{Solution :}}} \\
$\overrightarrow{v_1}=<-1,1,1> \\
\overrightarrow{v_2}=<2,2,-2> \\
\overrightarrow{v_1} \parallel \overrightarrow{v_2} \Rightarrow L_1 \parallel L_2 $ \\
\begin{exercise}
Pick a point on $L_1 ~~ t=0 \Rightarrow p(1,1,0)$
\end{exercise}
\end{example}
\begin{example}
If\\
$L_1 : x=1+t, ~~ ~~ y=2+3t , ~~ ~~ z=4-t \\ 
L_2 : x=2s , ~~ ~~ y=3+s, ~~  ~~ z=-3+4s$ \\ 
are skew , Find the distance between them .\\
{\color{smalt(darkpowderblue)}{\underline{Solution}}} : Construct two parallel planes that contain $L_1 \& L_2$ respectively \\
$P_1     
\hspace{7cm} 
P_2 
\\ 
point ~~t=0 \Rightarrow (1,-2,4) 
\hspace{2.8cm} 
point ~~ s=0 \Rightarrow (0,3,-3)\\
\overrightarrow{n}=\overrightarrow{v_1}*\overrightarrow{v_2} 
\hspace{5cm}
\overrightarrow{n}=<13,-6,-5>
\\
\overrightarrow{n}=13i-6j-5k 
\hspace{3.8cm}
13x-6y-5z+3=0 \\ 13x-6y-5z+-5=0 \\ d=-(13+12-20) \\
D=\cfrac{|13(0)-6(3)-5(-3)+-5|}{\sqrt{13^2+6^2+25}}=\cfrac{|-8|}{\sqrt{230}}=\cfrac{8}{\sqrt{230}}$
\end{example}
\noindent{\color{smalt(darkpowderblue)}\rule{\linewidth}{.2mm}}
\begin{example}
Determine whether each sentence is true or false .
\begin{enumerate}
    \item Two lines parallel to a third line are parallel .
    \item Two lines perpendicular to a third line are parallel .
    \item Two planes parallel to a third plane are parallel .
    \item Two planes perpendicular to a third plane are parallel .
    \item Two lines parallel to a plane are parallel.
    \item Two lines perpendicular to a plane are parallel .
    \item Two planes parallel to a line are parallel .
       \item Two planes perpendicular to a line are parallel .
       \item Two planes either intersect or are parallel .
       \item Two lines either intersect or are parallel .
       \item A plane and a line either intersect or are parallel .
    
\end{enumerate}
{\color{smalt(darkpowderblue)}{\underline{Solution}}}
\begin{enumerate}
\item T \hspace{2cm} 5. F \hspace{2cm} 9. T
\item F \hspace{2cm} 6. T \hspace{2cm} 10. F
\item T \hspace{2cm} 7. F \hspace{2cm} 11. T
\item F \hspace{2cm} 8. T  
\end{enumerate}
\end{example}
\noindent{\color{smalt(darkpowderblue)}\rule{\linewidth}{.2mm}}
\begin{example}
Show that the distance below the following planes \\
$P_1 : ax+by+cz+d_1=0 \\
P_2 : ax+by+cz+d_2=0 ,~~is  \\  D=\cfrac{|d_2-d_1|}{\sqrt{a^2+b^2+c^2}}$\\
{\color{smalt(darkpowderblue)}\underline{Solution :}} \\
Pick a point on $P_1(x_1,y_1,z_1)$ \\ So 
$D= {\cfrac{|ax_1+by_1+cz_1+d_2|}{\sqrt{a^2+b^2+c^2}} \\ = \cfrac{|d_2-d_1|}{\sqrt{a^2+b^2+c^2}}}$
\end{example}
\noindent{\color{smalt(darkpowderblue)}\rule{\linewidth}{.2mm}}
\begin{example}
Find equations of the parallel planes to \\ 
$P_1 : x-2y-2z+1=0 $ and units away from it.\\
{\color{smalt(darkpowderblue)}\underline{Solution :}} \\
$P_2 : x-2y-2z+d=0 \\ D=2 \\ {\cfrac{|d-1|}{\sqrt{1+4+4}} }=2\\ \Rightarrow |d-1|=6 \\
\Rightarrow  d-1 = -6 ~~~ OR~~~ d-1=6 \\
\Rightarrow d=-5 ~~~ OR ~~~ d=7$ 
\end{example}
\noindent{\color{smalt(darkpowderblue)}\rule{\linewidth}{.2mm}}
\begin{example}
Find the projection of the point
$(1,2,-1)$ on the plane \\$ 2x-2y+z-1=0 $\\
{\color{smalt(darkpowderblue)}\underline{Solution:}} \\
Let's construct a line that passes through (1,2,-1) $\&$ perpendicular to the plane .\\
point $(1,2,-1)$ \\
parallel vector $\overrightarrow{n}=<2,-2,1> \\
x=1+2t \\ 
y=2-2t \\
z=-1+t $ \\
we will find the intersection between the line and the plane \\
$2(1+2t)-2(2-2t)+(-1+t)-1=0 \\ 9t=4 \Rightarrow t=\cfrac{4}{9}$ \\
point $(1+\cfrac{8}{9} , 2-\cfrac{8}{9} , -1 + \cfrac{4}{9}) = (-\cfrac{17}{9},\cfrac{10}{9} , -\cfrac{5}{9})$
\end{example}
\noindent{\color{smalt(darkpowderblue)}\rule{\linewidth}{.2mm}}
\begin{example} 
\begin{enumerate}
    \item $x+y+z=c$
    \item $x+y+z=1$ 
    \item $(\cos{c})y+(\sin{c})z=1$
\end{enumerate} 
\end{example}
\noindent{\color{smalt(darkpowderblue)}\rule{\linewidth}{.2mm}}
\begin{problem}
1,3,5,9,11,12,13,14,17,19,21,25,26,29,30,31,35,37,38,39,45,46,48,51-57(odd),61-71(odd),74,76
\end{problem}
\chapter{Vector Function}
$f:\mathbb{R}\mapsto vector$ (real valued function)\\
$f(x)=sinx$\\
$f(\pi/2)=1$\\
\section{Vector Functions and space curres}
$\overrightarrow{r}(t):=<f(t),g(t),h(t)>\}$ vector function \\
$=f(t)i+g(t)j+h(t)k$\\
%----------------------------------------------------------------------------------
\noindent{\color{smalt(darkpowderblue)}\rule{\linewidth}{.2mm}}
\begin{example}
$\overrightarrow{r}(t)=<t^2,1-t,t^2+1>\\ \overrightarrow{r}(1)=<1,0,2>$
\end{example}
%--------------------------------------------------------------------------------
\noindent{\color{smalt(darkpowderblue)}\rule{\linewidth}{.2mm}}
\begin{example}
Find the domain of $\overrightarrow{r}(t)=<\sqrt{t},ln(1-t),t^2>$\\
$D_{\overrightarrow{r}(t)}=D_f\cap D_g\cap D_h=[0,1)$
\end{example}
\noindent{\color{smalt(darkpowderblue)}\rule{\linewidth}{.2mm}}
Limit and continuity :\\
if $\overrightarrow{r}(t)=<f(t),g(t),h(t)>$\\
then $\lim_{t\to t_\circ}\overrightarrow{r}(t)=<\lim_{t\to t_\circ}f(t),\lim_{t\to t_\circ}g(t),\lim_{t\to t_\circ}h(t)$\\
\noindent{\color{smalt(darkpowderblue)}\rule{\linewidth}{.2mm}}
\begin{example}
Find $\lim_{t\to t_\circ}<2t,\cfrac{\sin{t}}{\cfrac{1-e^t}{t}}>=<0,1,-1>$
\end{example}
%------------------------------------------------------------------------------
\noindent{\color{smalt(darkpowderblue)}\rule{\linewidth}{.2mm}}
$\overrightarrow{r}(t)$ is cont at $t_\circ\Leftrightarrow\lim_{t\rightarrow t_\circ}\overrightarrow{r}(t)=\overrightarrow{r}(t_\circ)$\\
\begin{definition}
Space Curve: suppose that $f,g,h$ are cont real-value function on I(interval) then \\
$x=f(t)$ , $y=g(t)$ , $z=h(t)$\\
is called space curve it can be represented using $\overrightarrow{r}(t)=<f(t),g(t),h(2)>$
\end{definition}
%-----------------------------------------------------------------------------
\begin{example}
Describe the curve defined by 
\begin{enumerate}
    \item $\overrightarrow{r}(t)=<t,1+t,2-t>$\\
    $x=t\\ y=1+t,~~~-\infty<t<\infty\\ z=2-t$\\
    \item $\overrightarrow{r}(t)=<\cos{t},\sin{t},t>$\\
    $x=\cos{t},~y=\sin{t},~z=t$
\end{enumerate}
\end{example}

\begin{definition}
Line Segment :the line segment from $\overrightarrow{r}_\circ$ to $\overrightarrow{r}_1,~0\leq t\leq 1$
\end{definition}
\noindent{\color{smalt(darkpowderblue)}\rule{\linewidth}{.2mm}}
%--------------------------------------------------------------------------
\begin{example}
Find the vector function that represented the line segment $P(1,2,-1)\&Q(2,3,2)$\\
$\overrightarrow{r}(t)=(1-t)<1,2,-1>+t<2,3,2>,~~~0\leq t \leq 1$\\
\end{example}
\noindent{\color{smalt(darkpowderblue)}\rule{\linewidth}{.2mm}}
\begin{example}
Find a vector function that represented the intersection of $x^2+y^2=1$ and $y+z=2$
\underline{\textbf{\large}\color{smalt(darkpowderblue)}Solution}\\
$\overrightarrow{r}(t)=<f(t),g(t),h(t)>$\\
$\overrightarrow{r}(t)=<\cos{t},\sin{t},2-\sin{t},~~~0\leq t\leq 1$\\
\end{example}
\noindent{\color{smalt(darkpowderblue)}\rule{\linewidth}{.2mm}}
\begin{problem}
1,3,5,15,17,25,27,35,37,42
\end{problem}
\section{13.2}
$\overrightarrow{r}'(t)=\lim_{h\to 0}\cfrac{\overrightarrow{r}(t+h)-\overrightarrow{r}(t)}{h}$\\
$=<f'(t),g'(t),h'(t)>$\\
$f'(t)i+g'(t)j+h'(t)k$\\
\noindent{\color{smalt(darkpowderblue)}\rule{\linewidth}{.2mm}}
\begin{example}
if $\overrightarrow{r}(t)=(1+t)i+te^{-t}j+\sin{2t}k$
\begin{itemize}
    \item[a] Find $\overrightarrow{r}'$
    \item[b] unit tangent vector at $t=0$
\end{itemize}
\underline{\textbf{\large}\color{smalt(darkpowderblue)}Solution}\\
$\overrightarrow{r}'(t)=2ti+(e^{-t}-te^{-t})j+2\cos{2t}k$\\
$\overrightarrow{r}(0)=<0,1,2>$ tangent vector \\
$\overrightarrow{T}(t)=\cfrac{\overrightarrow{r}(0)}{|\overrightarrow{r}'(0)|}=\cfrac{1}{\sqrt{0+1+4}}<0,1,2>=<0,\cfrac{1}{\sqrt{5}},\cfrac{2}{\sqrt{5}}>$
\end{example}
\noindent{\color{smalt(darkpowderblue)}\rule{\linewidth}{.2mm}}
\begin{example}
Find parametric equation for tangent line to the helix:\\
$\overrightarrow{r}(t)=<\cos{t},\sin{t},t>$ at $t=\pi$\\
\underline{\textbf{\large}\color{smalt(darkpowderblue)}Solution}\\
point :$(-1,0,\pi)$\\
parallel vector $\overrightarrow{r}'(t)=<-\sin{t},\cos{t},1>$\\
$\overrightarrow{r}'(\pi)=<0,-1,1>$\\
$x=-1+0t\\ y=0-t\\ z=\pi+t$
\end{example}
\noindent{\color{smalt(darkpowderblue)}\rule{\linewidth}{.2mm}}
%---------------------------------------------------------------------------
Differential Rule:
\begin{enumerate}
    \item $\cfrac{d}{dt}(\overrightarrow{u}(t)+\overrightarrow{v}(t))=\cfrac{d}{dt}\overrightarrow{u}(t)+\cfrac{d}{dt}\overrightarrow{v}(t)$
    \item $(c\overrightarrow{u}(t))'=c\overrightarrow{u}'(t)$
    \item $\cfrac{d}{dt}(f(t)\overrightarrow{u}(t))=f'(t)\overrightarrow{u}(t)+f(t)\overrightarrow{u}'(t)$
    \item $\cfrac{d}{dt}(\overrightarrow{u}(t).\overrightarrow{v}(t))=\overrightarrow{u},(t).\overrightarrow{v}(t)+\overrightarrow{u}(t).\overrightarrow{v}(t)$
    \item $\cfrac{d}{dt}(\overrightarrow{u}(t)\times\overrightarrow{v}(t))=\overrightarrow{u}'(t)\times\overrightarrow{v}(t)+\overrightarrow{u}t\times\overrightarrow{v}(t)$
    \item $\cfrac{d}{dt}\overrightarrow{u}(f(t))=f'(t)\overrightarrow{u}'(f(t))$
\end{enumerate}
\noindent{\color{smalt(darkpowderblue)}\rule{\linewidth}{.2mm}}
\begin{example}
if $\overrightarrow{u}'(1)\times\overrightarrow{v}(1)=<2,-1,3>$\\
$\overrightarrow{v}'(1)\times\overrightarrow{u}(1)=<2,-1,3>$\\
Find $(\overrightarrow{u}\times\overrightarrow{v})'(1)=\overrightarrow{u}'\times\overrightarrow{v}+\overrightarrow{u}\times\overrightarrow{v}$\\
$=<2,-1,3>+<-2,1,-3>=<0,0,0>$
\end{example}
\noindent{\color{smalt(darkpowderblue)}\rule{\linewidth}{.2mm}}
\begin{example}
if $|\overrightarrow{r}(t)|=c~~~(constant)$\\
show that $\overrightarrow{r}'(t)\bot\overrightarrow{r}(t)$\\
\underline{\textbf{\large}\color{smalt(darkpowderblue)}Proof}\\
$|\overrightarrow{r}(t)|=const$\\
$|\overrightarrow{r}(t)|^2=const$\\
$\overrightarrow{r}(t).\overrightarrow{r}(t)=const$
\end{example}
\noindent{\color{smalt(darkpowderblue)}\rule{\linewidth}{.2mm}}
Differentiate:\\
$\overrightarrow{r}'(t).\overrightarrow{r}(t)+\overrightarrow{r}(t).\overrightarrow{r}'(t)=0$\\
$2\overrightarrow{r}'(t).\overrightarrow{r}(t)=0$\\
$\overrightarrow{r}'(t).\overrightarrow{r}(t)=0$\\
$\overrightarrow{r}'(t)\bot \overrightarrow{r}(t)$
\begin{definition}
Integrals:
$\overrightarrow{r}(t)=<f(t),g(t),h(t)>$\\
then $\int\overrightarrow{r}(t)dt=<\int f(t)dt,\int g(t)dt,\int h(t)dt>$
\end{definition}
\noindent{\color{smalt(darkpowderblue)}\rule{\linewidth}{.2mm}}
\begin{example}
if $\overrightarrow{r}(t)=2ti-e^tj+lnt k$\\
Find $\overrightarrow{r}(t)$ where $\overrightarrow{r}(1)=<0,0,1>$
$\overrightarrow{r}(t)=<t^2+c_1',-e^t+c_2',tlnt-t+c_3'>$
$<0,0,1>\overrightarrow{r}(1)=<1+c_1,c_2-e,-1+c_3>$\\
$1+c_1=0\Rightarrow c_1=-1$\\
$c_2-e=0\Rightarrow c_2=e$\\
$-1+c_3=1\Rightarrow c_3=2$
\end{example}
\noindent{\color{smalt(darkpowderblue)}\rule{\linewidth}{.2mm}}
\begin{problem}
3,4,5,6,9,11,12,17,19,21,23,25,32,34,37,39,49
\end{problem}
\section{Arc Length}
$L=\int_{a}^{b}\sqrt{f'^2(t)+g'^2(t)+h'^2(t)}dt$\\
$L=\int_{a}^{b}=|\overrightarrow{r}'(t)|$\\
\noindent{\color{smalt(darkpowderblue)}\rule{\linewidth}{.2mm}}
\begin{example}
Find the length of the helix $\overrightarrow{r}(t)=<\cos{t},\sin{t},t>~~~~0\leq t\leq\pi$\\
\underline{\textbf{\large}\color{smalt(darkpowderblue)}Solution}\\
$L=\int_0^\pi \sqrt{sin^2t+cos^2t+1}=\int_0^\pi \sqrt{2}dt=\pi\sqrt{2}$
\end{example}
\noindent{\color{smalt(darkpowderblue)}\rule{\linewidth}{.2mm}}
\begin{example}
Two particles travel along the curves \\
$\overrightarrow{r}_1(t)=<t,t^2,t^3>$\\
$\overrightarrow{r}_2(t)=<1+2s,1+6s,1+14s>$
\begin{enumerate}
    \item Do the particles collide?
    \item Do their paths intersect?
\end{enumerate}
$t=1+2s,~~t^2=1+6s,~~t^3=1+14s\Rightarrow$\\
$(1+2s)^2=1+6s\Rightarrow1+4s+4s^2=1+6s\Rightarrow$\\
$s(4s-2)=0\Rightarrow s=0,s=1/2\Rightarrow$\\
$s=0,t=1\Rightarrow1\overset{?}{=}1$\\
$s=1/2,t=2\Rightarrow 8\overset{?}{=}8$\\
the paths intersect two lines, at the point $(1,1,1)~\&~(2,4,8)$\\
But they do not collide, the paths intersect at different $t,s$\\
\end{example}
\noindent{\color{smalt(darkpowderblue)}\rule{\linewidth}{.2mm}}
\begin{problem}
1,3,5
\end{problem}
\chapter{Partial Derivatives}
\section{Function of several variables}
\begin{definition}
A function of two variables is a rule that assigns for each $(x,y)$ in the domain one value $z=f(x,y)$ in the range \\
$x,y$ are called independent variables\\
$z$ is called an independent variable
\end{definition} 
\begin{example}
Find the domain of the following function 
\begin{enumerate}
    \item $f(x,y)=\sqrt{y-x+1}$\\
    $D=\{(x,y):y-x+1\geq 0\}$\\
    Range:$R=[0,\infty)$
    \item $f(x,y)=\sqrt{9-x^2-y^2}+\sqrt{x}$\\
    $D=\{(x,y):9-x^2-y^2\geq 0~\&~x\geq0\}$
    \item $f(x,y)=ln(x^2+y^2-9)$\\
    $D=\{(x,y):x^2+y^2-9>0$\\
    $R=(-\infty,\infty)$
    \item $f(x,y)=\cfrac{\sin^{-1}(x-y)}{\sqrt{x-y^2}}~~|~~\sin{[-\pi/2,\pi/2]}\to [-1,1]$\\
    $D=\{(x,y):-1\leq x-y\leq 1 ~\&~>0\}$
    \begin{enumerate}
        \item $-1\leq x-y$
        \item $x-y\leq 1$
        \item $x-y^2>0$
    \end{enumerate}
\end{enumerate}
\end{example}
\begin{example}
Find the range of $f(x,y)=\sqrt{9-x^2-y^2}$\\
$D=\{(x,y):x^2+y^2\leq 9$\\
$R=[0,3]$\\
$0\leq x^2+y^2\leq 9$\\
$0\geq -x^2-y^2\geq -9$\\
$9\geq 9-x^2-y^2\geq 0$\\
$3\geq \sqrt{9-x^2-y^2}\geq 0$\\
Domain 2D.\\
Range 1D.
\end{example}
\noindent{\color{smalt(darkpowderblue)}\rule{\linewidth}{.2mm}}

\begin{definition}
let $z=f(x,y)$ then the graph of the function is the set:\\
$G=\{(x,y,z):(x,y)\in D,z=f(x,y)$
\end{definition}
\begin{example}
Sketch the following 
\begin{enumerate}
    \item $f(x,y)=6-2x-3y$\\
    $z=6-2x-3y$\\
    $2x+3y+z-6=0$\\
    $x-intercept~~~y=0,z=0\Rightarrow x=3$
    $y-intercept~~~x=0,z=0\Rightarrow y=2$
    $z-intercept~~~x=0,y=0\Rightarrow z=0$
    \item $g(x,y)=\sqrt{9-x^2-y^2}$\\
    $z=\sqrt{9-x^2-y^2}$\\
    $z^2=9-x^2-y^2$\\
    $x^2+y^2+z^2=9,~z\geq0$
\end{enumerate}
\end{example}
\noindent{\color{smalt(darkpowderblue)}\rule{\linewidth}{.2mm}}
\begin{definition}
let $z=f(x,y)$ then the level curve of $f$ at $k\in $Range is the set 
$$L=\{(x,y):k=f(x,y)\}\supseteq\mathbb{R}^2$$
\end{definition}
\begin{example}
Find level curve to $f(x,y)=6-2x-3y$ at $k=0,6,-6...$\\
\underline{\textbf{\large}\color{smalt(darkpowderblue)}Solution}\\
 $k=0\Rightarrow 0=6-2x-3y\Rightarrow y=2-\cfrac{2}{3}x$\\
$k=6\Rightarrow 6=6-2x-3y\Rightarrow y=\cfrac{-2}{3}x$
\end{example}
\noindent{\color{smalt(darkpowderblue)}\rule{\linewidth}{.2mm}}
\begin{example}
Find level curve for $f(x,y)=\sqrt{9-x^2-y^2}$ at:\\ $k=1\Rightarrow1=\sqrt{9-x^2-y^2}\Rightarrow x^2+y^2=8$\\
$k=2\Rightarrow 2=\sqrt{9-x^2-y^2\Rightarrow x^2+y^2=5}$\\
$k=0\Rightarrow 0=\sqrt{9-x^2-y^2\Rightarrow x^2+y^2=9}$
\end{example}
\noindent{\color{smalt(darkpowderblue)}\rule{\linewidth}{.2mm}}
\begin{example}
Sketch some level curve of the function $f(x,y)=4x^2+y^2$\\
$k=1\Rightarrow 4x^2+y^2=1$\\
$\cfrac{x^2}{\cfrac{1}{4}}+y^2=1$\\
$k=4\Rightarrow 4x^2+y^2=4$\\
$x^2+\cfrac{y^2}{4}=1$
\end{example}
\noindent{\color{smalt(darkpowderblue)}\rule{\linewidth}{.2mm}}
\begin{definition}
Function of three variable \\
$\underset{dependent~variable}{w}=\underset{independent~variable}{f(x,y,z)}$\\
Domain: 3D\\
Range: 1D\\
Graph: 4D\\
Level surface:3D
\end{definition}
\begin{example}
Find the domain of $f(x,y,z)=\sqrt{9-x^2-y^2-z^2}$\\
$D=\{(x,y,z):x^2+y^2+z^2\leq9\}$
\end{example}
\noindent{\color{smalt(darkpowderblue)}\rule{\linewidth}{.2mm}}
\begin{example}
Find the following limits if exists:\\
\begin{enumerate}
    \item $\lim_{(x,y)\to (1,2)}(2x+y)=2(1)+2=4$
    \item $\lim_{(x,y)\to(0,0)}\cfrac{x^2-y^2}{x-y}=\lim_{(x,y)\to (0,0)}\cfrac{(x-y)(x+y)}{x_y}=0$
    \item $\lim_{(x,y)\to (0,0)}\cfrac{x^2-y^2}{x^2+y^2}$ we will take different paths.\\
    path 1:along $y=0$\\
    $\lim_{x\to 0}\cfrac{x^2}{x^2}=1$\\
    path 2:along $x=0$\\
    $\lim_{y\to 0}\cfrac{0-y^2}{0+y^2}=-1$\\
    $-1\neq 1\Rightarrow D.N.E$
    \item $\lim_{(x,y)\to (0,0)}\cfrac{xy}{x^2+y^2}$\\
    path $y=0\Rightarrow \lim_{x\to 0}\cfrac{0}{x^2}=0$\\
    path $x=0\Rightarrow \lim_{y\to0}\cfrac{0}{y^2}=0$\\
    path $y=x\Rightarrow \lim_{y\to 0}\cfrac{y^2}{y^2+y^2}=1/2$\\
    $0\neq 1/2\Rightarrow D.N.E$\\
    path $y=mx\Rightarrow\lim_{x\to 0}\cfrac{mx^2}{x^2+m^2x^2}=\cfrac{m}{1+m^2}$\\
    depends on $m$
    so D.N.E
\end{enumerate}
\end{example}
\begin{enumerate}
    \item[5.] $\lim_{(x,y) \to (0,0)} \cfrac{x y^{2}}{x^{2} + y^{4}}$\\
path $y = mx \rightarrow \lim_{x \to 0} \cfrac{x m^{2} x^{2}}{x^{2} + m^{4} x^{4}}$\\
$= \lim_{x\to 0} \cfrac{x^{3}}{x^{2}} \cfrac{m^{2}}{1+m^{4}x^{2}}$\\
$=\lim_{x\to 0}\cfrac{x^{3}}{x^{2}}\cfrac{m^{2}}{1+m^{4} x^{2}}$\\
              $  \rightarrow 0      \rightarrow m  $\\
              $=0$\\
 path $x = my^{2}$ \\
$\lim \cfrac{xy^{2}}{x^{2} + y^{2}} = \lim_{y \to 0} \cfrac{m y^{2} y^{2}}{m^{2} y^{4} + y^{4}} = \cfrac{m}{m^{2} +1}$ depends on m D.N.E
\item[6.] $\lim_{(x,y)\to(0,0)} \cfrac{sin(x^{2}+ y^{2})}{x^{2}+y^{2}}$\\
$r = \surd x^{2} + y^{2}$\\
$g= tan ^{-1} y      (x,y) \rightarrow 0 $\\
                      $r \rightarrow 0^{+}    g \rightarrow ?? $\\
 $= \lim_{r\to 0^{+}} \cfrac{sin r{2}}{{2}} = 1$
 \item[7.]$\lim_{(x,y) \to (0,0)} \cfrac{1-e^{\surd x^{2} +y^{2}}}{\surd x^{2} + y^{2}} = \lim_{r\to 0^{+}} \cfrac{1-e^{r}}{r} = \lim_{r\to 0^{+}} \cfrac{-e^{r}}{1} = -1 $
 \item[8.]$\lim_{(x,y)\to(0,0)} \cfrac{x^{3}-y^{3}}{x^{2} + y^{2}} = \lim_{r\to 0^{+}} \cfrac{r^{3} cos ^{3}\Theta - r^{3}sin{3}\Theta }{r^{2}}$\\
$=\lim_{r\to 0^{+}} r (cos^{3}\theta - sin^{3} \theta) = 0$
\item[9.]$\lim_{(x,y)\to (0,0)} \cfrac{xy}{x^{2} + y^{2}} = \lim_{r\to 0^{+}} \cfrac{r cos\theta r sin\theta }{r^{2}} = cos\theta sin\theta$ depends on $\theta$ dose not exist
\item[10.]$\lim_{(x,y) \to (0,0)} \cfrac{xy}{\surd x^{2} + y^{2}} = \lim_{r\to 0^{+}} \cfrac{r cos\theta r sin\theta}{r} = 0 $
\item[11.]$\lim_{(x,y) \to (0,0)} \cfrac{x^{2} + y^{2}}{x^{2} + y^{2}} = \lim\cfrac{r^{3} cos^{3}\theta + r^{3} sin^{3}\theta}{r^{2}(cos^{2}\theta-sin^{2}\theta}$\\
$=\lim_{r\to 0^{+}} r (\cfrac{cos^{3}\theta + sin^{3}\theta}{cos^{2}\theta - sin^{2}\theta}$ D.N.E / if $\theta = \cfrac{\pi}{y}$\\
continuity $z=f(x,y) is cont at (x_\cdot , y_\cdot)$ \\
if $\lim_{(x,y) \to (x_\cdot,y_\cdot)} f(x,y) = f(x_\cdot,y_\cdot)$
\end{enumerate}
\noindent{\color{smalt(darkpowderblue)}\rule{\linewidth}{.2mm}}
\begin{example}
find where the function 
$f(x,y) : \left\{ \begin{array}{cl}
\cfrac{x^{2} - y^{2}}{x^{2}+y^{2}} & :  (x,y) \neq (0,0) \\
0 & :  (x,y)=(0,0)\\
\end{array} \right.$, is cont\\ 
\underline{\textbf{\large}\color{smalt(darkpowderblue)}Solution}\\
$f$ is cont for $(x,y)\neq(0,0) at (0,0) ?$\\
\begin{itemize}
    \item $f(0,0) = 0$
    \item $\lim_{(x,y)\to(0,0)} \cfrac{x^{2}-y^{2}}{x^{2}+y^{2}}$ D.N.E $\rightarrow $ f is not cont at (0,0)
\end{itemize}
$f$ is cont $R^{2} \ {(0,0)}$
\end{example}
\noindent{\color{smalt(darkpowderblue)}\rule{\linewidth}{.2mm}}
\begin{problem}
6,7,9,11,13,15,19,20,21,23,25,29,41,61,63
\end{problem}
\section{Partial Derivatives}
\begin{definition}
if $z=f(x,y)$\\
$\cfrac{\partial z}{\partial x}=\cfrac{\partial f}{\partial x}= f_x = \lim_{h\to 0} \cfrac{f(x+h,y)-f(x,y)}{h}$\\
$\cfrac{\partial z}{\partial y}=\cfrac{\partial f}{\partial y}= f_y = \lim_{h\to 0} \cfrac{f(x,y+h)-f(x,y)}{h}$
\end{definition}

{\color{smalt(darkpowderblue)}Rules:}
\begin{enumerate}
\item to find $f_x$ treat y as a constant.
\item to find $f_y$ treat x as a constant.
\end{enumerate}
\noindent{\color{smalt(darkpowderblue)}\rule{\linewidth}{.2mm}}

\begin{example}
find $f_x , f_y$:
\begin{enumerate}
    \item $f(x,y) = x+y+xy $\\
$f_x = 1+0+y = 1+y $\\
$f_y = 1+x$
\item  $f(x,y) = \cfrac{x}{y}$\\
$f_x = \cfrac{1}{y}$\\
$f_y = x (\cfrac{-1}{y^{2}}$\\
3) $f(x,y) x e^{xy}$\\
$f_x = e^{xy} + xye^{xy}$\\
$f_y = x^{2}e^{xy}$\\
$f(x,y) = \cfrac{2}{x} -x\ln y$\\
$f_x = \cfrac{-2}{x^{2}} -\ln y$\\
$f_y = 0-\cfrac{x}{y}$
\end{enumerate}
\end{example}
\noindent{\color{smalt(darkpowderblue)}\rule{\linewidth}{.2mm}}
Function of three variables: \\
$w = f(x,y,z)$ to find \\
$f_x$ treat y,z as constant . \\
$f_y$ treat x,z as constant . \\
$f_z$ treat x,y as constant . \\
Ex : if $f(x,y,z) = xy +\cfrac{1}{z} - e^{xz}$\\
find $f_x = y+0+-z e^{xz}$\\
$f_y = x$\\
$f_z = 0+\cfrac{-1}{z^{2}} - xe^{xz}$\\
Higher Derivatives:
%رسمة

\noindent{\color{smalt(darkpowderblue)}\rule{\linewidth}{.2mm}}
\begin{example}
if $f(x,y) = x^{2} y + xy - 2ye^{x}$\\
find $f_{xx} , f_{xy} , f_{yy} $\\
$f_x = 2xy + y- 2ye^{x}$\\
$f_y = x^{2} + x- 2e^{x}$\\
$f_{xx} = 2y - 2ye^{x}$\\
$f_{xy} = 2x + 1 - 2e^{x}$\\
$f_{yx} = 2x+1-2e^{x} $\\
$f_{yy} = 0$\\
clairaut’s theorem : if f defined on a disc D that contains (a,b)
 if $f_{xy} , f_{yx} $ cont on D then $f_{xy}(a,b) = f_{yx}(a,b)$
\end{example}
\noindent{\color{smalt(darkpowderblue)}\rule{\linewidth}{.2mm}}

 \begin{example}
 T/F : there exists a function s.t 
 $f_x = 21x + 3y $\\
 $f_y = x^{2} - 2y$\\
 \underline{\textbf{\large}\color{smalt(darkpowderblue)}Solution:} note that 
 $f_{xy} = 3   f_{yx} = 2x $\\
 $f_{xy} \neq f_{yx} $ false .
 \end{example}
 \noindent{\color{smalt(darkpowderblue)}\rule{\linewidth}{.2mm}}

\begin{example}
if $f(x,y,z) = z^{2} cos(x+2y) $ find $f_{zxyz}$\\
\underline{\textbf{\large}\color{smalt(darkpowderblue)}Solution}
$f_z = 2z cos(x+2y)$\\
 $f_{zx} = -2z sin (x+2y)$\\
 $f_{zxy} = -4z cos(x+2y)$\\
 $f_{zxyz} = -ycos(x+2y)$
 \end{example}
 \noindent{\color{smalt(darkpowderblue)}\rule{\linewidth}{.2mm}}
{\color{smalt(darkpowderblue)}{Partial differential Equation:}}
\begin{enumerate}
    \item Laplace's Equation:
    $\cfrac{\partial^2 u}{\partial x^2}+\cfrac{\partial^2 u}{\partial y^2}=0$\\
    Solution of this equation are called :{\color{smalt(darkpowderblue)}harmonic equation}\\

    \begin{example}
    show that $u(x,y)=x^2-y^2$ is harmonic \\
    $u_x=2x$ , $u_y=-2y$\\
    $u_{xx}=2$, $u_{yy}=-2$\\
    so $u_{xx}+u_{yy}=0$
    \end{example}
    \noindent{\color{smalt(darkpowderblue)}\rule{\linewidth}{.2mm}}

    \begin{example}
    show that $u(x,y)=e^x\cos{y}$ satisfies Laplace's equation \\
    $u_x=e^x\sin{y}$ , $u_y=e^x\cos{y}$\\
    $u_{xx}=e^x\sin{y}$ , $u_{yy}=-e^x\sin{y}$\\
    $\Rightarrow u_{xx}+u_{yy}=0$
    \end{example}
    \noindent{\color{smalt(darkpowderblue)}\rule{\linewidth}{.2mm}}

    \item Wave equation \\
       \noindent\begin{minipage}{0.5\textwidth}
    $\cfrac{\partial^2 u}{\partial t^2}=a^2\cfrac{\partial^2 u}{\partial x^2}$\\
\end{minipage}
    \noindent\begin{minipage}{0.5\textwidth}
\begin{center}
   \includegraphics[width=6cm]{ss1.png}\\
\end{center}\end{minipage}
\end{enumerate}
    \begin{example}
    show that $u(x,t)=\sin{x-at}$ satisfies the wave equation \\
    $u_t=-acos(x-at)$ , $u_x=cos(x-at)$\\
    L.H.S=$u_{tt}=-a^2sin(x-at)$ , R.H.S=$u_{xx}=-sin(x-at)$\\
    L.H.S=R.H.S
    \end{example}
\noindent{\color{smalt(darkpowderblue)}\rule{\linewidth}{.2mm}}
\begin{problem}
5,7,8,9,12,13,14,1821,25,29,31,33,37,39,40
\end{problem}
\section{Tangent Plane and Linear Approximation}
Let $z=f(x,y)$ then the tangent plane at $(x_\circ,y_\circ)$\\
$z=z_\circ=f_x(x_\circ,y_\circ)(x-x_\circ)+f_y(x_\circ,y_\circ)(y-y_\circ)$\\
\noindent{\color{smalt(darkpowderblue)}\rule{\linewidth}{.2mm}}

\begin{example}
Find the tangent plane of $f(x,y)=2x^2+y^2$\\
at $(1,1,3)$\\
$f_x=4x\Rightarrow f_x(1,1)=4$\\
$f_y=2y\Rightarrow f_y(1,1)=2$\\
$z-3=4(x-1)+2(y-1)$\\
$4x+2y-z-3=0$
\end{example}
\noindent{\color{smalt(darkpowderblue)}\rule{\linewidth}{.2mm}}
Linear Approximation: let $z=f(x,y)$, $fx,fy$ cont\\
The linear approximation of $f$ at $(a,b)$\\
is $L(x,y)=f(a,b)+f_x(a,b)(x-a)+f_y(a,b)(y-b)$\\
\noindent{\color{smalt(darkpowderblue)}\rule{\linewidth}{.2mm}}

\begin{example}
let $f(x,y)=x e^{x y}$
\begin{enumerate}
    \item Find linear approximation at $(1,0)$ 
    \item Approximation $f(1.1,-0.1)$\\
\end{enumerate}
\underline{\textbf{\large}\color{smalt(darkpowderblue)}Solution}\\
$f(1,0)=1$\\
$f_x=xye^{xy}$\\
$f_x(1,0)=1$\\
$f_y=x^2e^{xy}$\\
$f_y(1,0)=1$\\
$\Rightarrow L(x,y)=1+1(x-1)+1(y-0)$\\
$L(x,y)=x+y\approx xe^{xy}$ around $(1,0)$\\
$f(1.1,-0.1)\approx L(1.1,-0.1)=1.1-0.1=1$\\
$f(1.1,-0.1)=0.9854$
\end{example}
\noindent{\color{smalt(darkpowderblue)}\rule{\linewidth}{.2mm}}

\begin{example}
if the tangent plane to $z=f(x,y)$ at $(2,3)$\\
is $2x-3y+z=1\Rightarrow L(x,y)=1-2x+3y$\\
Approximation $f(2.1,2.9)$\\
sol $f(2.1,2.9)\approx 1-2(2.1)+3(2.9)=1-4.2+8.7=5.5$
\end{example}
\noindent{\color{smalt(darkpowderblue)}\rule{\linewidth}{.2mm}}

\begin{example}
Approximation $12\sqrt{8.9}-12\sqrt[3]{8.1}$\\
\underline{\textbf{\large}\color{smalt(darkpowderblue)}Solution}\\
$f(x,y)=12\sqrt{x}-12\sqrt[3]{y}$ at $(9,8)$\\
$L(x,y)=12+2(x-9)-1(y-8)$\\
$f(8.9,8.1)\approx L(8.9,8.1)=12+2(\cfrac{-1}{10}+\cfrac{-1}{10}$\\
$12-\cfrac{3}{10}=11.7$
\end{example}
\begin{definition}
Differentials: if $ z=f(x,y) $, then we define the differential \\
$dz=f_x \partial x + f_y \partial y $\\
$dz=\cfrac{\partial f}{\partial x} \partial x+\cfrac{\partial f} {\partial y }\partial y$\\
let:\\
$\partial x = \delta x = x-a $\\
$ \partial y = \Delta y = y-a $\\
$ \partial z =\delta z = z - z\cdot = f(x,y) - f(a,b)$\\
$ dz = \cfrac{\partial f} {\partial x}\mid_{(a,b)} (x-a) + \cfrac{\partial f}{\partial y}\mid_{(a,b)}(y-b)$
\end{definition}
\noindent{\color{smalt(darkpowderblue)}\rule{\linewidth}{.2mm}}
%-------------------------------------------------------
\begin{example}
let $t=f(x,y)=x^{2}+3xy-y^{2}$
\begin{enumerate}
    \item find the differential\\
    $\partial t = (2x + 3y) \partial x + (3x - 2y) \partial y$\\
    \item if x change from 2 to 2.05\\
    y change from 3 to 2.96 \\
    com pane $\partial z , \delta z$
\end{enumerate}
$\partial z = (2(2) + 3(3)) \cfrac{5}{100} + (3.2.2.3)\cfrac{-4}{100}$\\
$=\cfrac{65}{100} = 0.65 $\\
$(2,3)\rightarrow (2.05,2.96)$\\
$\Delta z = f((2.05,2.96)-f(2,3)) = 0.6449 $\\
$dz = \delta z = z - z_\cdot$\\
\end{example}
\noindent{\color{smalt(darkpowderblue)}\rule{\linewidth}{.2mm}}
%-------------------------------------------------
functions of three variables\\
$w = f (x,y,z) $\\
$\partial w = f_x \partial x + f_y \partial y + f_z \partial z $\\
\noindent{\color{smalt(darkpowderblue)}\rule{\linewidth}{.2mm}}
%----------------------------------------------------
\begin{problem}
1-43(odd),44,48,49,50,51,53,59,61,65,71,72(a,d),73,75,77,87*,89*,93,94
\end{problem}
\section{Directional derivatives and Gradient vector}
$f_x=\lim_{h\to0} \cfrac{f(x+h,y)-f(x,y) }{h}$\\
$f_y = \lim_{h \to 0} \cfrac{f(x,y+h) - f(x,y)}{h}$\\
\noindent{\color{smalt(darkpowderblue)}\rule{\linewidth}{.2mm}}

\begin{definition}
the direction derivative of f at $(x_\cdot , y_\cdot)$ in the direction of the limit vector ${\overrightarrow{u}} = <a,b> is D_{\overrightarrow{u}} f(x_\cdot,y_\cdot)=\lim\cfrac{f(x+a h,y+b h) - f(x,y)}{h}$
\end{definition} 

Gradient vector:\\
\begin{definition}
if $z = f(x,y) , then the gradient of f at (x_\cdot,y_\cdot) is \nabla f = < f_x (x_\cdot ,y_\cdot),f_y(x_\cdot , y_\cdot)>$
\end{definition}
%------------------------------------------------------
\noindent{\color{smalt(darkpowderblue)}\rule{\linewidth}{.2mm}}
\begin{example}
if $ f(x,y) = x^{2} - y^{2} find \nabla f\mid_(1,2)$\\
$f_x = 2x \rightarrow f_x(1,2) = 2 $\\
$f_y = 2y \rightarrow f_y(1,2) = -4$\\
\underline{\textbf{\large}\color{smalt(darkpowderblue)}Solution}\\,$\nabla f\mid_(1,2) = <2,-4>$
\end{example}

%----------------------------------------------------
\begin{theorem}
 $D_{\overrightarrow{u}} f(x_\cdot,y_\cdot) = \nabla f .{\overrightarrow{u}}$
\end{theorem}
%-----------------------------------------------------
\begin{example}
if $f(x,y) = sin x + e ^{xy}$ find the directional of f at (0,1) in the direction of ${\overrightarrow{u}} = <3,-4>$\\
so, $D_{\overrightarrow{u}} f(0,1)= \nabla f .{\overrightarrow{u}}$\\
$=<2,0> . <\cfrac{3}{5},\cfrac{4}{5}>\\
=\cfrac{6}{10} + 0\cfrac{4}{5} = (0,6)$\\
$\nabla f = <f_x,f_y> . <2,0>\\
f_x = cos x + y e^{xy} \\
f_y = x e^{xy} \\
f_x(0,1) = 2\\
f_y(0,1) = 0 $\\
function of three variables $w = f(x,y,z)$\\
$\nabla f = ,f_x,f_y,f_z>$\\
$D_{\overrightarrow{u}} f(a,b,c) = \nabla f . {\overrightarrow{u}}$\\
\end{example}
\noindent{\color{smalt(darkpowderblue)}\rule{\linewidth}{.2mm}}
%----------------------------------------------------------------------
\begin{example}
find the direction derivative of $f(x,y,z) = \cfrac{x - y}{z} + x^{2} + e^{y}$\\
at $p(1,0,1) in the directional to the point Q (-1,2,0)$\\
\underline{\textbf{\large}\color{smalt(darkpowderblue)}Solution}\\ : $D_{\overrightarrow{u}} f(-1,2,0) = \nabla f . {\overrightarrow{u}} = <3,0,-1>.<\cfrac{2}{3},\cfrac{2}{3},\cfrac{-1}{3}> = -2 + 0 + \cfrac{1}{3} = -1\cfrac{2}{3}$\\
${\overrightarrow{u}} = \cfrac{{\overrightarrow{PQ}}}{|{\overrightarrow{PQ}}|} = <\cfrac{-2}{3} , \cfrac{2}{3} , \cfrac{-1}{0}>$\\
$f_x = \cfrac{1}{z} + 2x , f_x(1,0,1) = 1+2 = 3$\\
$f_y = \cfrac{-1}{z} + e^{y} \rightarrow f_y(91,0,1) = -1 +1 = 0 $\\
$f_y = \cfrac{y-x}{z^{2}} , f_y(1,0,1) = \cfrac{0-1}{1^{2}} = -1 $\\
$\nabla f = <3,0,-1>$\\
\end{example}
\noindent{\color{smalt(darkpowderblue)}\rule{\linewidth}{.2mm}}
%------------------------------------------------------
\begin{example}
if $D_{\overrightarrow{u}} f = 3    \nabla 2f .{\overrightarrow{u}}$\\
if $D_{\overrightarrow{u}} 2f = 6$\\
$D_{\overrightarrow{u}} -2f = -6$\\
$D_{\overrightarrow{2u}} f = 3$\\
$D_{\overrightarrow{-2u}} f = -3 $
\end{example}
\noindent{\color{smalt(darkpowderblue)}\rule{\linewidth}{.2mm}}
Question $z = f(x,y)(x_\cdot , y_\cdot)$\\
${\overrightarrow{u}} = ??$\\
find ${\overrightarrow{u}}$ that maximize $D_{\overrightarrow{u}} f(x_\cdot , y_\cdot = \nabla f .{\overrightarrow{u}} = |\nabla f| |{\overrightarrow{u}}| cos$\\


\begin{theorem}
 \begin{enumerate}
    \item the max directional derivative of f at $(x_\cdot , y_\cdot) is |\nabla f|$ and it accrues if${\overrightarrow{u}}$ has the same direction of $\nabla f$ .
    \item the min directional derivative of f is $-|\nabla f | $ and it accrues if${\overrightarrow{u}}$ has the opposite direction of $\nabla f$.
\end{enumerate}
\end{theorem}

\begin{example}
Let $z = f(x.y) = xe^{y}$ find the max directional derivative at (2,0).\\
\underline{\textbf{\large}\color{smalt(darkpowderblue)}Solution}\\
$\nabla f = <f_x ,f_y> , f_x = e^{y} f_x(2,0) = 1 , f_y = xe^{y} f_y (2,0) = 2$\\
$\nabla f = <1,2>$\\
$max D_{\overrightarrow{u}} f(2,0) = |\nabla f| = \surd 5$\\
it occurs if ${\overrightarrow{u}}$ has the same direction of $<1,2>$\\
max directional derivative $\leftrightarrow$ max rate of change \\
                           $\leftrightarrow$ increasing most rapidly \\
min directional derivative $\leftrightarrow$ min rate of change\\
                           $\leftrightarrow$ decreasing most rapidly

\end{example} 
\noindent{\color{smalt(darkpowderblue)}\rule{\linewidth}{.2mm}}
%===================================================
\underline{\textbf{\large}\color{smalt(darkpowderblue)}Tangent plane for level surfaces}\\
\noindent\begin{minipage}{0.5\textwidth}
$k = f(x,y,z)$ level surfaces\\
to find the plane we need 
    1.point $(x_\cdot , y_\cdot , z_\cdot)$
    2.${\overrightarrow{n}} = \nabla f$\\
\end{minipage}
\noindent\begin{minipage}{0.5\textwidth}
\begin{center}
   \includegraphics[width=5cm]{rr1.png}\\
\end{center}\end{minipage}
\noindent{\color{smalt(darkpowderblue)}\rule{\linewidth}{.2mm}}

\begin{example}
\begin{enumerate}
    \item find the equation of the tangent plane to $\cfrac{x^{2}}{4} + y^{2} + \cfrac{z^{2}}{9} = 3 at (2,1,3)$
    \item find the equation of the normal liner
\end{enumerate}

\underline{\textbf{\large}\color{smalt(darkpowderblue)}Solution}
\begin{enumerate}
    \item plane ! point (2,1.3) \\
${\overrightarrow{n}} = \nabla f = <\cfrac{2x}{4},2y,\cfrac{2}{9} = <1,2,\cfrac{2}{3}$ \\
$|(x-2) + 2(y-1) + \cfrac{2}{3} (z-3)|$
\item point (2,1,3)\\
${\overrightarrow{n}} = <1,2,\cfrac{2}{3}> , x=2+t \\
                                          , y=1+2t\\
                                          , z=3+\cfrac{2}{3}t $\\
$\nabla f = <f_x,f_y,f_z>$\\
$D_{\overrightarrow{u}} f(x_\cdot , y_\cdot , z_\cdot) = \nabla f .{\overrightarrow{u}}$\\
max $D_{\overrightarrow{u}} f = |\nabla f|$ it accrues if $\nabla f ,{\overrightarrow{u}}$ have the same direction \\
F(x,y,z) = K the n the normal to the tangent $D_{\overrightarrow{n}} f = |\nabla f|$ 
\end{enumerate}
\end{example}
\noindent{\color{smalt(darkpowderblue)}\rule{\linewidth}{.2mm}}
\begin{problem}
1,3,5,6,11,13,15,17,19,21,25,30,31,33,35,36,37
\end{problem}
\section{The Chain Rule}
\noindent\begin{minipage}{0.5\textwidth}
\begin{center}
   \includegraphics[width=3cm]{case1.png}\\
\end{center}
\end{minipage}
\noindent\begin{minipage}{0.5\textwidth}
Case I:\\
$\cfrac{dz}{dt}=\cfrac{\partial  z}{\partial  x}\cfrac{dx}{dt}+\cfrac{\partial  z}{\partial  y}\cfrac{dy}{dt}$\\
\end{minipage}
\noindent\begin{minipage}{0.5\textwidth}
\begin{center}
   \includegraphics[width=5cm]{case2.png}\\
\end{center}\end{minipage}
\noindent\begin{minipage}{0.3\textwidth}
Case II:\\
$\cfrac{\partial  z}{\partial  t}=\cfrac{\partial  z}{\partial  x}\cfrac{\partial  x}{\partial  t}+\cfrac{\partial  z}{\partial  y}\cfrac{\partial  y}{\partial  t}$\\
$\cfrac{\partial  t}{\partial s}=\cfrac{\partial z}{\partial x}\cfrac{\partial x}{\partial s}+\cfrac{\partial z}{\partial y}\cfrac{\partial y}{\partial s}$\\
\end{minipage}
\\
\noindent\begin{minipage}{0.5\textwidth}
\begin{center}
   \includegraphics[width=6cm]{case3.png}\\
\end{center}\end{minipage}
\noindent\begin{minipage}{0.3\textwidth}
Case III:\\
$\cfrac{\partial w}{\partial s}=\cfrac{\partial w}{\partial x}\cfrac{\partial x}{\partial s}+\cfrac{\partial w}{\partial y}\cfrac{\partial y}{\partial s}+\cfrac{\partial w}{\partial z}\cfrac{\partial z}{\partial s}$\\
\end{minipage}\\
%-------------------------------------------------------
\noindent{\color{smalt(darkpowderblue)}\rule{\linewidth}{.2mm}}
\begin{example}
If $u=x^4 y+y^2 z^3$\\
\noindent\begin{minipage}{0.5\textwidth}
$x=r s e^t$\\
$y=rs^2 e^{-t}$\\
$z=r^2 s \sin{t}$\\
Find $\cfrac{\partial u}{\partial s}$ when $r=2,~s=1,~t=0$\\
\end{minipage}
\noindent\begin{minipage}{0.5\textwidth}
\begin{center}
   \includegraphics[width=6cm]{case3.png}\\
\end{center}\end{minipage}
\underline{\textbf{\large}\color{smalt(darkpowderblue)}Solution}\\
$\cfrac{\partial u}{\partial s}=u_x\cfrac{\partial x}{\partial s}+u_y\cfrac{\partial y}{\partial s}+u_z\cfrac{\partial z}{\partial s}$\\
$=(4x^3 y)r e^t+(x^4+2y z^3)(2r s e^{-t})+(3y^2t^2)(r^2\sin{t})$\\
$\cfrac{\partial u}{\partial s}|_{(r,s,t)=(2,1,0)}=(64)2+(16+0)4+0=128+64=192$
\end{example}
%-------------------------------------------------------
\noindent{\color{smalt(darkpowderblue)}\rule{\linewidth}{.2mm}}
\begin{example}.\\
\noindent\begin{minipage}{0.5\textwidth}
if $g(s,t)=f(\overset{x}{s^2-t^2},\overset{y}{t^2-s^2})$\\
show that $g$ satisfies $t\cfrac{\partial g}{\partial s}+s\cfrac{\partial g}{\partial t}=0$\\
\end{minipage}
\noindent\begin{minipage}{0.5\textwidth}
\begin{center}
   \includegraphics[width=5cm]{case2.png}\\
\end{center}\end{minipage}
\underline{\textbf{\large}\color{smalt(darkpowderblue)}Solution}\\
$g(s,t)=f(x,y)$\\
$\cfrac{\partial g}{\partial s}=\cfrac{\partial g}{\partial x}\cfrac{\partial x}{\partial s}+\cfrac{\partial g}{\partial y}\cfrac{\partial y}{\partial s}$\\
$\cfrac{\partial g}{\partial s}=f_x(2s)+f_y(-2s)\cdots 1$\\
$\cfrac{\partial g}{\partial t}=\cfrac{\partial g}{\partial x}\cfrac{\partial x}{\partial t}+\cfrac{\partial g}{\partial y}\cfrac{\partial y}{\partial t}$\\
$\cfrac{\partial g}{\partial t}=f_x(-2t)+f_y(2t)\cdots 2$\\
so, $t\cfrac{\partial g}{\partial s}+s\cfrac{\partial g}{\partial t}=f_x(2st)+f_y(-2st)+f_x(-2st)+f_y(2st)=0$\\
\end{example}
%--------------------------------------------------------
\noindent{\color{smalt(darkpowderblue)}\rule{\linewidth}{.2mm}}
\begin{example}.\\
\noindent\begin{minipage}{0.5\textwidth}
let $z=f(x,y)$ , $(f$ has cont second partial derivative)\\
if $x=r^2+s^2$ , $y=2rs$\\
find:\\
1. $\cfrac{\partial z}{\partial r}$\\
2. $\cfrac{\partial^2z}{\partial r^2}$\\
\end{minipage}
\noindent\begin{minipage}{0.5\textwidth}
\begin{center}
   \includegraphics[width=5cm]{case22.png}\\
\end{center}\end{minipage}
\underline{\color{smalt(darkpowderblue)}Solution}\\
\noindent\begin{minipage}{0.5\textwidth}
$\cfrac{\partial z}{\partial r}=f_x\cfrac{\partial x}{\partial r}+f_y\cfrac{\partial y}{\partial r}=f_x(2r)+f_y(2s)$\\
$=2rf_x+2sf_y$\\
\end{minipage}
\noindent\begin{minipage}{0.5\textwidth}
\begin{center}
   \includegraphics[width=3.5cm]{case23.png}\\
\end{center}\end{minipage}
\noindent\begin{minipage}{0.5\textwidth}
$\cfrac{\partial^2 z}{\partial r^2}=2[r(f_{xx}\cfrac{\partial x}{\partial r}+f_{x y}\cfrac{\partial y}{\partial r})+f_x]+2s[f_{y x}\cfrac{\partial x}{\partial r}+f_{y y}\cfrac{\partial y}{\partial r}]$\\
$=(4r^2)f_{xx}+8rs f_{x y}+4s^2f_{y y}$\\
\end{minipage}
\noindent\begin{minipage}{0.5\textwidth}
\begin{center}
   \includegraphics[width=3.5cm]{case24.png}\\
\end{center}\end{minipage}
\end{example}
\noindent{\color{smalt(darkpowderblue)}\rule{\linewidth}{.2mm}}
%-----------------------------------------------------
\begin{definition}
Implicit differentiation\\
if $F(x,y,z)=0$\\
then $\cfrac{\partial z}{\partial x}=-\cfrac{\cfrac{\partial F}{\partial x}}{\cfrac{\partial F}{\partial z}}$\\
$\cfrac{\partial z}{\partial y}=-\cfrac{\cfrac{\partial F}{\partial y}}{\cfrac{\partial F}{\partial z}}$
\end{definition}
%--------------------------------------------------
\noindent{\color{smalt(darkpowderblue)}\rule{\linewidth}{.2mm}}
\begin{example}
if $x^3+y^3+z^3=1-6x y z$\\
Find $\cfrac{\partial z}{\partial x},\cfrac{\partial z}{\partial y}$\\
\underline{\color{smalt(darkpowderblue)}Solution}\\
$x^3+y^3+z^3+6xyz-1=0$\\
$\cfrac{\partial z}{\partial x}=-\cfrac{F_x}{F_z}=-\cfrac{3x^2+6yz}{3z^2+6xy}$\\
$\cfrac{\partial z}{\partial y}=-\cfrac{F_y}{F_z}=-\cfrac{3y^2+6xz}{3z^2+6xy}$
\end{example}
\noindent{\color{smalt(darkpowderblue)}\rule{\linewidth}{.2mm}}
\begin{problem}
3,4,5,7,9,11,12,13,14,15,16,17,19,21,23,26,27,28,31,33,35,39,43,46,48,50,51,58
\end{problem}